% Standalone LaTeX file for Table 1: Summary Statistics
% To compile: pdflatex table1_summary_stats_standalone.tex
\documentclass[12pt]{article}

% Packages
\usepackage[utf8]{inputenc}
\usepackage{booktabs}
\usepackage{caption}
\usepackage{graphicx}
\usepackage{subcaption}
\usepackage{threeparttable}
\usepackage[margin=1in]{geometry}
\usepackage{pdflscape}
\usepackage{comment}
\newcommand{\sym}[1]{\rlap{$^{#1}$}}

\begin{document}
Written by Minseon Park, 12/07,25. List of figure and tables. 

\section{Figures}

\begin{figure}[htbp]
\centering
\caption{Bargaining Index and Perceived Bargaining Power}
\label{fig:figure1}
\begin{subfigure}{0.48\textwidth}
	\centering
	\includegraphics[width=\linewidth]{../results/ccei_bargaining_whose_suggestion.png}
\end{subfigure}
\begin{subfigure}{0.48\textwidth}
	\centering
	\includegraphics[width=\linewidth]{../results/ccei_bargaining_had_individual.png}
\end{subfigure}
\end{figure}

\begin{figure}[htbp]
\caption{Correlation Across Measures}\label{fig:figure1}
\includegraphics[width=\linewidth]{../results/correlation_heatmap.png}
\end{figure}

\newpage
\section{Tables}

\begin{table}[htbp]\centering
\caption{Summary Statistics and Balance Test}
\label{tab:summary}
\begin{tabular}{lccc}
\hline \toprule 
 & (1) & (2) & (3) \\
Outcome Variable & Mean & SD & Baseline-Endline Correlation \\
\midrule
\multicolumn{4}{l}{\textit{Panel A: Individual-Level Measures from the Experiment}} \\
CCEI &  0.900 &  0.133 &  0.195** \\
Risk Attitude &  0.322 &  0.134 &  0.392** \\
Bargaining Power Index &  0.500 &  0.354 &  0.155** \\
N & 1304 & &  \\
\midrule
\multicolumn{4}{l}{\textit{Panel B: Individual-Level Other Characteristics}} \\
Male &  0.608 &  0.488 & \\
Height & 163.536 &  7.750 & \\
\textit{Friendship Network:} & & & \\
\quad Out-Degree &  4.630 &  2.535 &  0.557** \\
\quad In-Degree &  4.623 &  2.692 &  0.688** \\
 N & 1304 & & \\
Math Score &  2.657 &  1.515 &  0.523** \\
\textit{Big 5 Personality:} & & & \\
\quad Outgoing &  3.570 &  1.006 & \\
\quad Opened &  3.568 &  0.865 & \\
\quad Agreeable &  2.848 &  0.744 & \\
\quad Conscientious &  3.386 &  0.859 & \\
\quad Stable &  2.500 &  0.810 & \\
N & 1286 & & \\
\midrule
\multicolumn{4}{l}{\textit{Panel C: Group-Level Measures from the Experiment}} \\
CCEI &  0.912 &  0.141 &  0.243** \\
Risk Attitude &  0.297 &  0.142 &  0.395** \\
N  & 652 & & \\
\midrule
\multicolumn{4}{l}{\textit{Panel D: Group-Level Friendship}} \\
Friendship: One-sided &  0.178 &  0.383 &  0.221** \\
Friendship: Mutual &  0.130 &  0.337 &  0.554** \\
N & 652 & &  \\
\bottomrule \hline
\end{tabular}

\begin{tablenotes}
\small
\item \textit{Notes:} This table presents summary statistics at baseline. In-degree is the number of classmates who nominated the student as a friend. Out-degree is the number of classmates the student nominated as friends.
The correlation column shows the correlation between baseline and endline values 
for variables we measured both in the baseline and the endline experiment. + $p<$ 0.1, * $p<$ 0.05, ** $p<$ 0.01. 
\end{tablenotes}
\end{table}

\begin{comment}
\begin{landscape}
\begin{table}[htbp]\centering
\caption{Correlation Across Measures}
\label{tab:summary}
\begin{tabular}{l*{5}{c}}
\hline \toprule
  & (1) & (2) & (3) & (4) & (5) \\
  & CCEI & Risk Attitude & Bargaining Power Index & Out-Degree & In-Degree \\
\midrule
\multicolumn{6}{l}{\textit{Experimental Measures:}} \\
\quad CCEI &  &  &  &  &  \\
\quad Risk Attitude & -0.225** &  &  &  &  \\
\quad Bargaining Power Index & -0.410** & 0.154** &  &  &  \\
\multicolumn{6}{l}{\textit{Friendship Network:}} \\
\quad Out-Degree & -0.028 & -0.008 & -0.004 &  &  \\
\quad In-Degree & -0.070* & -0.063* & -0.009 & 0.338** &  \\
\multicolumn{6}{l}{\textit{Demographics:}} \\
\quad Male & -0.082** & -0.225** & -0.011 & 0.075** & 0.054+ \\
\quad Height & -0.016 & -0.164** & -0.020 & -0.022 & 0.066* \\
\multicolumn{6}{l}{\textit{Cognitive Score:}} \\
\quad Math Score & 0.048+ & -0.035 & -0.088** & 0.066* & 0.169** \\
\quad RAT Score & 0.043 & 0.010 & -0.027 & 0.017 & 0.115** \\
\multicolumn{6}{l}{\textit{Big 5 Personality:}} \\
\quad Outgoing & -0.056* & -0.068* & -0.040 & 0.036 & 0.127** \\
\quad Opened & -0.002 & -0.068* & -0.053+ & 0.039 & 0.044 \\
\quad Agreeable & -0.001 & 0.047+ & 0.007 & 0.023 & 0.019 \\
\quad Conscientious & 0.012 & 0.018 & -0.041 & 0.023 & 0.030 \\
\quad Stable & 0.057* & 0.025 & 0.012 & -0.010 & -0.052+ \\
\bottomrule \hline
\end{tabular}

% Notes: ** p<0.01, * p<0.05, + p<0.1
% Lower triangle shows correlations with key variables

\begin{tablenotes}
\small
\item \textit{Notes:} This table presents correlation among variables at baseline. In-degree is the number of classmates who nominated the student as a friend. Out-degree is the number of classmates the student nominated as friends. + $p<$ 0.1, * $p<$ 0.05, ** $p<$ 0.01. 
\end{tablenotes}
\end{table}
\end{landscape}
\end{comment}

\begin{table}[htbp]
\center
\begin{tabular}{lccccc}
\toprule \hline
& (1) & (2) & (3) & (4) & (5) \\
Survey Characteristics & No & Yes & Yes & Yes & Yes  \\
Risk Attitude & No & No & Yes & Yes & Yes \\
Individual FE & No & No & No & No & Yes \\ \midrule
High CCEI       &   -0.447\sym{**}&   -0.438\sym{**}&   -0.426\sym{**}&   -0.396\sym{**}&   -0.403\sym{**}\\
                &  (0.016)        &  (0.016)        &  (0.018)        &  (0.025)        &  (0.022)        \\
Math Score      &                 &   -0.000        &   -0.000        &   -0.000        &   -0.001        \\
                &                 &  (0.000)        &  (0.000)        &  (0.000)        &  (0.001)        \\
Diff in Math    &                 &   -0.013\sym{*} &   -0.013\sym{*} &   -0.013\sym{*} &    0.002        \\
                &                 &  (0.005)        &  (0.005)        &  (0.005)        &  (0.007)        \\
Risk Attitude   &                 &                 &    0.001        &    0.001        &   -0.001        \\
                &                 &                 &  (0.001)        &  (0.001)        &  (0.001)        \\
Diff in RA      &                 &                 &    0.333\sym{**}&    0.332\sym{**}&    0.277\sym{**}\\
                &                 &                 &  (0.057)        &  (0.057)        &  (0.083)        \\
High CCEI*Post  &                 &                 &                 &   -0.062\sym{+} &                 \\
                &                 &                 &                 &  (0.035)        &                 \\
N               &     2228        &     2228        &     2228        &     2228        &     2228        \\
R-squared       &    0.382        &    0.394        &    0.420        &    0.422        &    0.738        \\
 
\\ \hline \bottomrule
\end{tabular}
\end{table}


\begin{table}[htbp]
\center
\begin{tabular}{lccccc}
\toprule \hline
& (1) & (2) & (3) & (4) & (5) \\
Survey Characteristics & No & Yes & Yes & Yes & Yes  \\
Risk Attitude & No & No & Yes & Yes & Yes \\
Individual FE & No & No & No & No & Yes \\ \midrule
High CCEI       &   -0.116\sym{**}&   -0.101\sym{**}&   -0.092\sym{**}&   -0.052        &   -0.090\sym{*} \\
                &  (0.025)        &  (0.024)        &  (0.025)        &  (0.033)        &  (0.037)        \\
Math Score      &                 &    0.181        &    0.204        &    0.204        &    0.319        \\
                &                 &  (0.159)        &  (0.159)        &  (0.159)        &  (0.244)        \\
Diff in Math    &                 &   -0.098        &   -0.110        &   -0.110        &   -0.161        \\
                &                 &  (0.079)        &  (0.079)        &  (0.079)        &  (0.124)        \\
Risk Attitude   &                 &                 &    3.607        &    3.607        &    6.397\sym{+} \\
                &                 &                 &  (2.211)        &  (2.212)        &  (3.237)        \\
Diff in RA      &                 &                 &   -1.424        &   -1.426        &   -2.871\sym{+} \\
                &                 &                 &  (1.107)        &  (1.106)        &  (1.609)        \\
High CCEI*Post  &                 &                 &                 &   -0.083\sym{+} &                 \\
                &                 &                 &                 &  (0.048)        &                 \\
N               &     2216        &     2216        &     2216        &     2216        &     2204        \\
R-squared       &    0.059        &    0.065        &    0.073        &    0.073        &    0.509        \\
 
\\ \hline \bottomrule
\end{tabular}
\end{table}


\begin{table}[htbp]
\center
\begin{tabular}{lccccc}
\toprule \hline
& (1) & (2) & (3) & (4) & (5) \\
Survey Characteristics & No & Yes & Yes & Yes & Yes  \\
Risk Attitude & No & No & Yes & Yes & Yes \\
Group FE & No & No & No & No & Yes \\ \midrule
$\text{CCEI}\_{\text{max},gt}$&    0.362\sym{***}&    0.350\sym{***}&                  &    0.330\sym{**} \\
                &  (0.083)         &  (0.099)         &                  &  (0.096)         \\
$\text{CCEI}\_{\text{dist},gt}$&   -0.234\sym{***}&   -0.271\sym{***}&                  &   -0.265\sym{***}\\
                &  (0.046)         &  (0.058)         &                  &  (0.058)         \\
Endline         &                  &    0.028         &                  &   -0.002         \\
                &                  &  (0.186)         &                  &  (0.186)         \\
$\text{Endline} \times \text{CCEI}\_{\text{max},gt}$ &                  &   -0.027         &                  &    0.003         \\
                &                  &  (0.187)         &                  &  (0.188)         \\
$\text{Endline} \times \text{CCEI}\_{\text{dist},gt}$&                  &    0.078         &                  &    0.078         \\
                &                  &  (0.074)         &                  &  (0.075)         \\
$\text{Math Score}\_{\text{max},gt}$&                  &                  &    0.010\sym{*}  &    0.006         \\
                &                  &                  &  (0.004)         &  (0.004)         \\
$\text{Math Score}\_{\text{dist},gt}$&                  &                  &   -0.003         &   -0.001         \\
                &                  &                  &  (0.003)         &  (0.003)         \\
inclass\_n\_friends\_max&                  &                  &    0.005\sym{*}  &    0.004\sym{*}  \\
                &                  &                  &  (0.002)         &  (0.002)         \\
inclass\_n\_friends\_dist&                  &                  &   -0.006\sym{*}  &   -0.005\sym{*}  \\
                &                  &                  &  (0.003)         &  (0.003)         \\
N               &     1304         &     1304         &     1304         &     1304         \\
R-squared       &    0.166         &    0.169         &    0.119         &    0.191         \\
 
\\ \hline \bottomrule
\end{tabular}
\end{table}

\appendix
\setcounter{table}{0}
\setcounter{figure}{0}
\renewcommand{\thetable}{A\arabic{table}}
\renewcommand{\thefigure}{A\arabic{figure}}

\newpage
\section{Appendix Figures}

\begin{figure}[htbp]
\centering
\caption{Bargaining Index and Perceived Bargaining Power By Within-Pair Risk Attitude Difference}
\begin{subfigure}{0.48\textwidth}
	\centering
	\includegraphics[width=\linewidth]{../results/ccei_bargaining_whose_suggestion_high_RA_diff.png}
	\caption{High RA difference}
	\label{fig:fig1_left_high}
\end{subfigure}
\begin{subfigure}{0.48\textwidth}
	\centering
	\includegraphics[width=\linewidth]{../results/ccei_bargaining_whose_suggestion_low_RA_diff.png}
	\caption{Low RA difference}
	\label{fig:fig1_left_low}
\end{subfigure}
\hfill
\begin{subfigure}{0.48\textwidth}
	\centering
	\includegraphics[width=\linewidth]{../results/ccei_bargaining_had_individual_high_RA_diff.png}
	\caption{High RA difference}
	\label{fig:fig1_right_high}
\end{subfigure}
\hfill
\begin{subfigure}{0.48\textwidth}
	\centering
	\includegraphics[width=\linewidth]{../results/ccei_bargaining_had_individual_low_RA_diff.png}
	\caption{Low RA difference}
	\label{fig:fig1_right_low}
\end{subfigure}
\end{figure}


\end{document}
