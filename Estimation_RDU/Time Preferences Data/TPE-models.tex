
\documentclass[12pt]{article}
%%%%%%%%%%%%%%%%%%%%%%%%%%%%%%%%%%%%%%%%%%%%%%%%%%%%%%%%%%%%%%%%%%%%%%%%%%%%%%%%%%%%%%%%%%%%%%%%%%%%%%%%%%%%%%%%%%%%%%%%%%%%%%%%%%%%%%%%%%%%%%%%%%%%%%%%%%%%%%%%%%%%%%%%%%%%%%%%%%%%%%%%%%%%%%%%%%%%%%%%%%%%%%%%%%%%%%%%%%%%%%%%%%%%%%%%%%%%%%%%%%%%%%%%%%%%
\usepackage{amsfonts}
\usepackage{mitpress}

%TCIDATA{OutputFilter=LATEX.DLL}
%TCIDATA{Version=5.50.0.2953}
%TCIDATA{<META NAME="SaveForMode" CONTENT="1">}
%TCIDATA{BibliographyScheme=Manual}
%TCIDATA{Created=Tuesday, June 28, 2011 11:27:27}
%TCIDATA{LastRevised=Wednesday, November 16, 2011 18:05:11}
%TCIDATA{<META NAME="GraphicsSave" CONTENT="32">}
%TCIDATA{<META NAME="DocumentShell" CONTENT="Articles\SW\A Simple MIT Press Article">}
%TCIDATA{CSTFile=LaTeX Article (bright).cst}

\newtheorem{theorem}{Theorem}
\newtheorem{acknowledgement}[theorem]{Acknowledgement}
\newtheorem{algorithm}[theorem]{Algorithm}
\newtheorem{axiom}[theorem]{Axiom}
\newtheorem{case}[theorem]{Case}
\newtheorem{claim}[theorem]{Claim}
\newtheorem{conclusion}[theorem]{Conclusion}
\newtheorem{condition}[theorem]{Condition}
\newtheorem{conjecture}[theorem]{Conjecture}
\newtheorem{corollary}[theorem]{Corollary}
\newtheorem{criterion}[theorem]{Criterion}
\newtheorem{definition}[theorem]{Definition}
\newtheorem{example}[theorem]{Example}
\newtheorem{exercise}[theorem]{Exercise}
\newtheorem{lemma}[theorem]{Lemma}
\newtheorem{notation}[theorem]{Notation}
\newtheorem{problem}[theorem]{Problem}
\newtheorem{proposition}[theorem]{Proposition}
\newtheorem{remark}[theorem]{Remark}
\newtheorem{solution}[theorem]{Solution}
\newtheorem{summary}[theorem]{Summary}
\newenvironment{proof}[1][Proof]{\noindent\textbf{#1.} }{\ \rule{0.5em}{0.5em}}
\newdimen\dummy
\dummy=\oddsidemargin
\addtolength{\dummy}{72pt}
\marginparwidth=.5\dummy
\marginparsep=.1\dummy
\input{tcilatex}
\begin{document}


\section{Decision problem}

An individual subject in the time preference experiment chooses a
consumption profile over two periods, $\left( c_{t},c_{t+s}\right) $, to
maximize his intertemporal utilities subject to a budget constraint:

\[
u\left( c_{t}|\omega _{t}\right) +\beta _{t,s}u\left( c_{t+s}|\omega
_{t+s}\right) 
\]%
\[
\text{s.t.}\ c_{t}+\frac{1}{1+r_{t,s}}c_{t+s}=m, 
\]%
where $u\left( \cdot |\omega \right) $ is the per-period utility function
based on the baseline consumption level, $\omega $. $\beta _{t,s}$ denotes a
discount factor. If we assume the exponential discounting function, the
discount factor only depends on the time horizon between two periods, $s$. $%
r_{t,s}$ represents an experimental interest rate of payments between two
periods.

\section{Optimal demands}

We derive the optimal demands in each of exponential and power utility
functions.

\subsection{Exponential utility function}

Assume that the utility function $u$ is represented by the exponential
function:%
\[
u\left( c|\omega \right) =-\frac{1}{A}\exp \left( -A\left( c-\omega \right)
\right) \text{,} 
\]%
where $A$ determines the curvature of utility function, which is an absolute
risk aversion coefficient in the decision making under risk. Note that in
this case the optimal demand does not depend on the baseline consumption
level, $\omega $. From the constrained maximization problem, we have the
following results.

\subparagraph{Case 1: $-m\leq \frac{1}{A}\ln \left( \protect\beta %
_{t,s}\left( 1+r_{t,s}\right) \right) \leq \left( 1+r_{t,s}\right) m$}

\begin{eqnarray*}
c_{t}^{\ast } &=&\frac{1}{2+r_{t,s}}\left[ \left( 1+r_{t,s}\right) m-\frac{1%
}{A}\ln \left( \beta _{t,s}\left( 1+r_{t,s}\right) \right) \right] , \\
c_{t+s}^{\ast } &=&\frac{1+r_{t,s}}{2+r_{t,s}}\left[ m+\frac{1}{A}\ln \left(
\beta _{t,s}\left( 1+r_{t,s}\right) \right) \right] \text{.}
\end{eqnarray*}

\subparagraph{Case 2: $\frac{1}{A}\ln \left( \protect\beta _{t,s}\left(
1+r_{t,s}\right) \right) <-m$}

\[
c_{t}^{\ast }=m\text{ \ and \ }c_{t+s}^{\ast }=0\text{.} 
\]

\subparagraph{Case 3: $\frac{1}{A}\ln \left( \protect\beta _{t,s}\left(
1+r_{t,s}\right) \right) >\left( 1+r_{t,s}\right) m$}

\[
c_{t}^{\ast }=0\text{ \ and \ }c_{t+s}^{\ast }=\left( 1+r_{t,s}\right) m%
\text{.} 
\]

\subsection{Power utility function}

The utility function $u$ is assumed to be represented by the power utility
function:%
\[
u\left( c|\omega \right) =\frac{\left( c-\omega \right) ^{1-\rho }}{1-\rho }%
\text{,} 
\]%
where $\rho $ represents a relative risk aversion coefficient. Note that the
optimal demand depends on the basline consumption level in this case and
also that, for $\omega \leq 0$, the optimality of demands can not generate
corner solutions, widely observed in the experimental data. Thus, we assume
that $\omega >0$.

\subparagraph{Case 1: $0<c_{t}^{\ast }<m$ and $0<c_{t+s}^{\ast }<\left(
1+r_{t,s}\right) m$}

\begin{eqnarray*}
c_{t}^{\ast } &=&\frac{1}{\left( 1+r_{t,s}\right) +K}\left[ \left(
1+r_{t,s}\right) m+\left( K-1\right) \omega \right] , \\
c_{t+s}^{\ast } &=&\frac{K}{\left( 1+r_{t,s}\right) +K}\left[ \left(
1+r_{t,s}\right) m+\left( K-1\right) \omega \right] -\left( K-1\right)
\omega \text{,}
\end{eqnarray*}%
where%
\[
K=\left[ \beta _{t,s}\left( 1+r_{t,s}\right) \right] ^{1/\rho }\text{.} 
\]

\subparagraph{Case 2 \& 3:}

\begin{eqnarray*}
c_{t}^{\ast } &=&m\text{ \ and \ }c_{t+s}^{\ast }=0, \\
&&\text{or} \\
c_{t}^{\ast } &=&0\text{ \ and \ }c_{t+s}^{\ast }=\left( 1+r_{t,s}\right) m%
\text{.}
\end{eqnarray*}

\section{Estimation}

We estimate $\theta =$ $\left( A,\beta _{t,s}\right) $ or $\left( \rho
,\beta _{t,s}\right) $ using the individual-level experimental data. Each
treatment consists of 50 decision problems for each individual, denoted by $%
\left\{ \left( c_{t}^{p},c_{t+s}^{p},m^{p},r_{t,s}^{p}\right) \right\}
_{p=1}^{50}$. Our econometric method is to minmize the distance between
observed demands and optimal demands:%
\[
D\left( \theta \right) =\sum_{p=1}^{50}\left[ \left( c_{t}^{p}-c_{t}^{p\ast
}\left( \theta \right) \right) ^{2}+\left( c_{t+s}^{p}-c_{t+s}^{p\ast
}\left( \theta \right) \right) ^{2}\right] \text{.} 
\]

\end{document}
