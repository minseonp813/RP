\documentclass[10pt,a4paper]{article}
\usepackage[utf8]{inputenc}
\usepackage{amsmath}
\usepackage{amsfonts}
\usepackage{amssymb}
\author{Minseon Park}
\begin{document}
\large\textbf{2.2 Power Utility Function}
\begin{equation*}
u(c|w)=\frac{(c+w)^{(1-\rho)}}{1-\rho} 
\end{equation*}
\textbf{Case 1.} $c_t>0$ and $c_{t+s}=0$, which means $\lambda_{t+s}>0$

F.O.Cs are 
\begin{equation}
(c_t+w)^{-\rho}-\lambda=0
\end{equation}
\begin{equation}
\beta(c_{t+s}+w)^{-\rho}-\frac{1}{1+r}\lambda+\lambda_{t+s}=0
\end{equation}

\begin{align*}
\lambda_{t+s}>0 \\
&\iff -\beta w^{-\rho} + \frac{1}{1+r}(m+w)^{-\rho} >0 \\
&\iff \frac{1}{1+r}>\frac{\beta(m+w)^{\rho}}{w^{\rho}} \\
&\iff 1+r < \frac{w^{\rho}}{\beta(m+w)^{\rho}}
\end{align*}


\textbf{Case 2.} $c_t=0$ and $c_{t+s}>0$, which means $\lambda_t>0$

F.O.Cs are 
\begin{equation}
(c_t+w)^{-\rho}-\lambda+\lambda_t=0
\end{equation}
\begin{equation}
\beta(c_{t+s}+w)^{-\rho}-\frac{1}{1+r}\lambda=0
\end{equation}

As in case 1, we can get 
\begin{equation}
(1+r)>\frac{((1+r)m+w)^{\rho}}{w^{\rho}\beta}
\end{equation}


\textbf{Case 3.}
F.O.Cs are 
\begin{equation}
(c_t+w)^{-\rho}-\lambda=0
\end{equation}
\begin{equation}
\beta(c_{t+s}+w)^{-\rho}-\frac{1}{1+r}\lambda=0
\end{equation}

(6) and (7) yields
\begin{equation}
c_t+w=((1+r)\beta)^{-\frac{1}{\rho}}(c_{t+s}+w)
\end{equation}

Using the budget constraint,

\begin{equation*}
c^*_{t+s}=\frac{K(1+r)}{K+(1+r)}\lbrace m+(1-\frac{1}{K})w \rbrace
\end{equation*}
\begin{equation*}
c^*_t=\frac{(1+r)}{K+(1+r)}m+\frac{1-K}{K+(1+r)}w
\end{equation*}
where $K=((1+r)\beta)^{\frac{1}{\rho}}$

\end{document}