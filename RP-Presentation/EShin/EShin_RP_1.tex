\documentclass[10pt, xcolor={dvipsnames,table}]{beamer}
%\documentclass[11pt,handout]{beamer} % for handout or editing

\usetheme{Madrid}
\usecolortheme{dolphin} % Simple and clean template
\setbeamertemplate{footline}[frame number]
\usefonttheme{professionalfonts}

% reduce the margins
\setbeamersize{text margin left = 0.25 in}
\setbeamersize{text margin right = 0.25 in}

%
\usepackage{graphicx}
\usepackage{amssymb}
\usepackage{amsfonts}
\usepackage{amsmath}
\usepackage{amsthm}
\usepackage{mathrsfs}
\usepackage{color}
\usepackage{setspace}
\usepackage{bbm}
\usepackage{subfigure}

%without auto-caption labeling
\usepackage{stackengine}

%%%%% for graphics in latex (e.g. arrows)
\usepackage{tikz}
\usetikzlibrary{arrows,chains,matrix,positioning,scopes,automata}
\tikzset{
  invisible/.style={opacity=0},
  visible on/.style={alt={#1{}{invisible}}},
  alt/.code args={<#1>#2#3}{%
    \alt<#1>{\pgfkeysalso{#2}}{\pgfkeysalso{#3}} % \pgfkeysalso doesn't change the path
  },
}

%%%%% draw functions by using tikz
\usepackage{pgfplots}

%colors
\definecolor{tred}{rgb}{0.35, 0.0, 0.0}
\definecolor{tblue}{rgb}{0.0, 0.0, 0.35}
\definecolor{grey}{rgb}{0.4, 0.4, 0.4}
\definecolor{wbgrey}{rgb}{0.9, 0.9, 1}
\definecolor{wrgrey}{rgb}{1, 0.9, 0.9}
\definecolor{wggrey}{rgb}{0.9, 1, 0.9}
\definecolor{periwinkle}{rgb}{0.8, 0.8, 1.0}
\definecolor{purpleheart}{rgb}{0.41, 0.21, 0.61}

%block color
\setbeamercolor{block title}{bg=purpleheart!90, fg=white}

\usepackage{mathtools}
\usepackage[export]{adjustbox}

% set separated appendix numbers
\usepackage{appendixnumberbeamer}

%%% vertical skips
\newcommand{\sk}{\vspace{0.1 in}}
\newcommand{\mk}{\vspace{0.2 in}}
\newcommand{\bk}{\vspace{0.3 in}}
\newcommand{\nk}{\vspace{- 0.1 in}}


\DeclarePairedDelimiter{\ceil}{\lceil}{\rceil}

\newcommand{\scrG}{\mathscr{G}}
\newcommand{\scrF}{\mathscr{F}}

\newtheorem{proposition}[theorem]{Proposition}

% turn off navigation symbols
\setbeamertemplate{navigation symbols}{}

% for appendix
\usepackage{appendixnumberbeamer}

%\usepackage{bm}         % For typesetting bold math (not \mathbold)
%\logo{\includegraphics[height=0.6cm]{caltechslide.pdf}}

%%%%%%%%%%%%%%%%%%%%%%%%%%%%%%%% Here are some custumized commands %%%%%%%%%%%%%%%%%%%%%%%%%%%%%%%%%%%%%%

\newtheorem{axiom}{Axiom}

\newcommand{\plim}{\overset{p}{\longrightarrow}}% converges in probability to.
\newcommand{\rthlim}{\overset{L^r}{\longrightarrow}}% converges in r-th mean
\newcommand{\ip}[1]{\langle #1 \rangle}
\newcommand{\Lim}[1]{\raisebox{0.5ex}{\scalebox{0.8}{$\displaystyle \lim_{#1}\;$}}}

\newcommand{\argmax}{\operatornamewithlimits{argmax}}
\newcommand{\argmin}{\operatornamewithlimits{argmin}}
\newcommand{\bigtimes}{\operatornamewithlimits{\BIGOP{\times}}}

\newcommand{\pr}{\mathbbm{P}}
\newcommand{\ex}{\mathbbm{E}}
\newcommand{\norm}[1]{\left|\left|#1\right|\right|}

% Number systems
\newcommand{\NN}{\mathbb{N}}
\newcommand{\ZZ}{\mathbb{Z}}
\newcommand{\QQ}{\mathbb{Q}}
\newcommand{\RR}{\mathbb{R}}

\newcommand{\GG}{\bm{G}}
\newcommand{\RN}{\bm{N}}
\newcommand{\RM}{\bm{M}}
\newcommand{\RZ}{\bm{Z}}
\newcommand{\bI}{\bm{I}}
\newcommand{\XX}{\bm{X}}

\newcommand{\ff}{\bm{f}}
\newcommand{\xx}{\bm{x}}
\newcommand{\ba}{\bm{a}}
\newcommand{\rz}{\bm{z}}


\newcommand{\idc}{\mathbbm{1}}

\newcommand{\calE}{\mathcal{E}}
\newcommand{\calP}{\mathcal{P}}
\newcommand{\calR}{\mathcal{R}}
\newcommand{\calF}{\mathcal{F}}
\newcommand{\calV}{\mathcal{V}}
\newcommand{\calI}{\mathcal{I}}

\newcommand{\rmu}{\bm{\mu}}
\newcommand{\Rm}{\bm{m}}
\newcommand{\pp}{\bm{p}}
\newcommand{\dd}{\bm{d}}

\newcommand{\DD}{\mathcal{D}}

\definecolor{tred}{RGB}{200,50,80}
\definecolor{tgreen}{RGB}{80,180,80}
\definecolor{tblue}{RGB}{50,50,250}
\definecolor{tpupple}{RGB}{150,50,200}

% reduce space around align
\usepackage{etoolbox}
\newcommand{\zerodisplayskips}{%
  \setlength{\abovedisplayskip}{0.05 in}
  \setlength{\belowdisplayskip}{0.05 in}
  \setlength{\abovedisplayshortskip}{0.05 in}
  \setlength{\belowdisplayshortskip}{0.05 in}}
\appto{\normalsize}{\zerodisplayskips}
\appto{\small}{\zerodisplayskips}
\appto{\footnotesize}{\zerodisplayskips}

\setbeamerfont{caption}{size=\scriptsize}
\setbeamerfont{footnote}{size=\scriptsize}


% include pdf version 1.6
%\pdfoptionpdfminorversion=7 


\usepackage{kotex}
\usepackage{textcomp} % this is to use Korean currency, Won.

\usepackage{scrextend}

\usepackage{fancybox}     
\usepackage{booktabs}              
\usepackage{multirow}

\newcolumntype{?}{!{\vrule width 1pt}}

%%%% testing environments

%\usepackage{thmtools}
%\usepackage{boiboites} % this requires boiboites.sty in the folder
\usepackage{eshinbox}

\newcounter{thm}
\numberwithin{thm}{section}

\newboxedtheorem[boxcolor=black, background=gray!5, titlebackground=gray!20,
titleboxcolor = black]{thm}{Theorem}{thm}

\newboxedtheorem[boxcolor=black, background=gray!5, titlebackground=gray!20,
titleboxcolor = black]{prp}{Proposition}{thm}

%%%%%%%%%%%%%%%%%%%%%%%%%%%%%%%% START %%%%%%%%%%%%%%%%%%%%%%%%%%%%%%%%%%%%%%

\title[Short title of the talk]{Rationality and Preference Aggregation of Group Decision under Risk}
\author{Syngjoo Choi (SNU), Booyuel Kim (KDIS), Minseon Park (Wisconsin), Yoonsoo Park (KDI), Euncheol Shin (KHU)}
\institute[]
{Presenter: Euncheol Shin (Kyung Hee University) \\
\medskip
{\tt{eshin.econ@khu.ac.kr}}
}
\date{%June, 2017
\today %will show current date.
%324 Alternatively, you can specify a date.
}

%
\begin{document}
%
\begin{frame}
\titlepage
\end{frame}

%%%%%%%%%%%%%%%%%%%%%%%% new frame %%%%%%%%%%%%%%%%%%%%%%%%

\begin{frame}{Introduction}

\begin{itemize}

\item In various contexts, many important decisions are made by groups.
\mk

\item Individuals in the collective are \textcolor{tred}{\textbf{heterogeneous}}.
\sk
	\begin{itemize}
	\item Rationality
	\sk
	
	\item Time preference: household savings and consumption decisions
	\sk
	
	\item Risk preference: risk assessment in environmental policy-making in committees
	
	\end{itemize}
	\mk
	
\item The expansion of democratic institutions and rapid progress in communication technology (e.g., Skype and SNS) highlight the importance of exploring the process of group decisions.

\end{itemize}

\end{frame}

%%%%%%%%%%%%%%%%%%%%%%%% new frame %%%%%%%%%%%%%%%%%%%%%%%%

\begin{frame}{Introduction: Why Experiments?}

\begin{itemize}

\item The laboratory experiment can be stripped of many confounding factors, and decisions can be observed in a highly \textcolor{tred}{\textbf{controlled environment}}.
\mk

\item We can directly measure rationality and risk preference without noise.
\mk
\pause

\item Beyond measuring risk preference, we can investigate how risk preferences are \textcolor{tred}{\textbf{aggregated}}.
\mk

\item We measure \textcolor{tred}{\textbf{rationality}} of both individuals and group decisions by applying the \textcolor{tred}{\textbf{revealed preference theory}}.
\mk

\item We finally analyze efficiency of group decisions.

\end{itemize}

\end{frame}

%%%%%%%%%%%%%%%%%%%%%%%% new frame %%%%%%%%%%%%%%%%%%%%%%%%

\begin{frame}{Introduction: Research Questions}

\begin{itemize}

\item[1.] Rationality extension:
	\sk
	\begin{itemize}
	
	\item If each individual's choices are consistent with a utility maximization model, do a group's choices also tend to be?	
	
	\end{itemize}
	\mk

\item[2] Risk preference aggregation:
	\sk
	\begin{itemize}
	
	\item Are individual's risk preferences reflected into that of a group?
	
	\end{itemize}
	\mk
	
\item[3.] Efficiency:
	\sk
	\begin{itemize}
	
	\item Are a group's choices Pareto efficient? Is it maximizing social welfare?
	
	\end{itemize}
	
\end{itemize}

\end{frame}

%%%%%%%%%%%%%%%%%%%%%%%% new frame %%%%%%%%%%%%%%%%%%%%%%%%

\begin{frame}{Related Literature}

\begin{itemize}

\item[1.] Preference aggregation
	\sk
	\begin{itemize}
	
	\item If each individual's choices are consistent with a utility maximization model, do a group's choices also tend to be?	
	
	\end{itemize}
	\mk

\item[2] Testable implication
	\sk
	\begin{itemize}
	
	\item Are individual's risk preferences reflected into that of a group?
	
	\end{itemize}
	\mk
	
\item[3.] Intra-household bargaining
	\sk
	\begin{itemize}
	
	\item Are a group's choices Pareto efficient? Is it maximizing social welfare?
	
	\end{itemize}
	
\end{itemize}

\end{frame}

%%%%%%%%%%%%%%%%%%%%%%%% new frame %%%%%%%%%%%%%%%%%%%%%%%%

\begin{frame}[noframenumbering]{}

\center
\textbf{Experimental Design and Subjects}

\end{frame}

%%%%%%%%%%%%%%%%%%%%%%%% new frame %%%%%%%%%%%%%%%%%%%%%%%%

\begin{frame}{Experimental Design}
\nk

\begin{figure}
\center
\large 
\begin{tikzpicture}

% axes
\draw[->, thick] (0,0) -- (7,0) node[anchor = west] {$x_r$};
\draw[->, thick] (0,0) -- (0,7) node[anchor = east] {$x_b$};
		
% budget lines
\draw[thick] (6,0) -- (0,3);

% 45 degree line
\draw[dotted, thick] (0,0) -- (4,4) node[anchor = south west] {$x_r = x_b$};
\draw (0.8, 0) node[anchor = south] {$45^\circ$};

\visible<2>{
% indifference curve
\draw[dotted, thick, tred] (6,0) -- (0,6) node[anchor = south west] {};
\draw (5, 2) node[anchor = south] {\small{\textcolor{tred}{\textbf{indifference curve}}}};
}

% prices
\draw[->, thick] (6,0) -- (6+1/2, 1) node[anchor = west] {$p = (p_r, p_b)$};

% bundles
\draw (2, 2) node {\textbullet};
\draw (2, 2.1) node[anchor = south] {$x$};

\draw (3, 6-3/2*3) node {\textbullet};
\draw (3, 6-3/2*3) node[anchor = north east] {$x'$};

\draw (6, 0) node {\textbullet};
\draw (6, - 0.6) node[anchor = south] {$x''$};

% texts
\visible<1>{
\small{
\draw (3, 7) node[anchor = west] {Two equally likely states: $\textcolor{tred}{R}$ and $\textcolor{tblue}{B}$.};

\draw (3, 6.5) node[anchor = west] {There are two associated Arrow securities.};

\draw (3, 6) node[anchor = west] {$x_r$ is the demand for the security that pays off in state $\textcolor{tred}{R}$.};

\draw (3, 5.5) node[anchor = west] {$x_b$ is the demand for the security that pays off in state $\textcolor{tblue}{B}$.};

\draw (3, 5) node[anchor = west] {Budget constraint: $p_r x_r + p_b x_b = 1$.};
}
}

\visible<2>{
\small{
\draw (3, 7) node[anchor = west] {In this example, $p_b > p_r$.};

\draw (3, 6.5) node[anchor = west] {Risk neutral agent will choose $x''$.};

\draw (3, 6) node[anchor = west] {Extremely risk averse agent will choose $x$.};

\draw (3, 5.5) node[anchor = west] {Intermediately risk averse agent will choose $x'$.};

%\draw (3, 5) node[anchor = west] {$r = \frac{x_r}{x_r + x_b}$};
}
}

\end{tikzpicture}
\end{figure}

\end{frame}

%%%%%%%%%%%%%%%%%%%%%%%% new frame %%%%%%%%%%%%%%%%%%%%%%%%

\begin{frame}{Procedure and Subjects}

\begin{itemize}

\item We conducted the experiment in 12 middle schools in Daegu.
\mk

\item The number of students: 
\mk 

\item The instructions were read aloud by an experiment in each classroom.
\mk
	
\item Each subject participate in two sessions: individual and group.
\mk
	
\item Each session consisted of 18 independent decision rounds.
\mk

\item Each round started by having the computer select a budget line randomly from the set of lines that intersect at least one axis at or above $300$ KRW or below $1500$ KRW.

\end{itemize}

\end{frame}

%%%%%%%%%%%%%%%%%%%%%%%% new frame %%%%%%%%%%%%%%%%%%%%%%%%

\begin{frame}{Procedure and Subjects}

\begin{itemize}

\item At the end of each round, the computer randomly selected one of the two states ($\textcolor{tred}{R}$ and $\textcolor{tblue}{B}$).
\mk

\item Subjects were not informed of the state that was actually selected at the end of each round.
\mk

\item Each subject was paid for he/she earned in a randomly selected round.
\mk

\item After finishing the first session, students start the second session.
\mk

\item Two students in the same classroom were randomly matched.
\mk

\item A student moved to the other partner's desk and shared the computer.

\end{itemize}

\end{frame}

%%%%%%%%%%%%%%%%%%%%%%%% new frame %%%%%%%%%%%%%%%%%%%%%%%%

\begin{frame}{Experimental Design}

\begin{figure}
\center
\includegraphics[width = 4.0 in]{figures/pblrealshot}
\end{figure}

\end{frame}

%%%%%%%%%%%%%%%%%%%%%%%% new frame %%%%%%%%%%%%%%%%%%%%%%%%

\begin{frame}{Experimental Design: Screenshot}

\begin{figure}
\center
\includegraphics[width = 3.8 in]{figures/expshot}
\end{figure}

\end{frame}

%%%%%%%%%%%%%%%%%%%%%%%% new frame %%%%%%%%%%%%%%%%%%%%%%%%

\begin{frame}[noframenumbering]{Experimental Design: Example}

\begin{figure}
\center
\includegraphics[width = 4.2 in]{figures/Figure_Choice_Examples}
\end{figure}

\end{frame}

%%%%%%%%%%%%%%%%%%%%%%%% new frame %%%%%%%%%%%%%%%%%%%%%%%%

\begin{frame}[noframenumbering]{}

\center
\textbf{Result 1: Rationality Extension}

\end{frame}

%%%%%%%%%%%%%%%%%%%%%%%% new frame %%%%%%%%%%%%%%%%%%%%%%%%

\begin{frame}{Measurement: Afriat's Efficiency Index (a.k.a. CCEI)}

\begin{figure}
\center
\large 
\begin{tikzpicture}

% axes
\draw[->, thick] (0,0) -- (7,0) node[anchor = west] {};
\draw[->, thick] (0,0) -- (0,7) node[anchor = east] {};
		
% budget lines
\draw[thick] (6,0) -- (0,4);
\draw[thick] (0,6) -- (4,0);

% bundles
\draw (3.3, 6-3/2*3.3) node {\textbullet};
\draw (3.3, 6-3/2*3.3) node[anchor = north east] {$x'$};

\draw (6-3/2*3, 3) node {\textbullet};
\draw (6-3/2*3, 3) node[anchor = north east] {$x$};

% rationalization
\draw[red, ultra thick] (0,6-3/2*0.5) -- (4-0.5,0);

% texts
\small{
\draw (3, 7) node[anchor = west] {If there are two goods, a choice dataset satisfies the GARP};

\draw (3, 6.5) node[anchor = west] {if and only if it satisfies the WARP.};

\draw (3, 5.5) node[anchor = west] {$x'$ and $x$ violate GARP.};

\draw (3, 4.5) node[anchor = west] {If the budget line for $x'$ is deflated, then GARP is satisfied.};

\draw (3, 4) node[anchor = west] {Choose $e_v$ for each violation $v$.};

\draw (3, 3) node[anchor = west] {CCEI is defined as the supremum over all the numbers $e_v$'s.};

}

\end{tikzpicture}
\end{figure}

\end{frame}

%%%%%%%%%%%%%%%%%%%%%%%% new frame %%%%%%%%%%%%%%%%%%%%%%%%

\begin{frame}
\frametitle{Measurement: Afriat's Efficiency Index (a.k.a. CCEI)}

\begin{itemize}

\item By definition, CCEI $\in [0,1]$.
\mk

\item The bigger CCEI is, the less severe violation of GARP.
\mk
	
\item \textbf{Research Question:} 
\mk

\item[] Individual Rationality $\uparrow$ $\Rightarrow$ Group Rationality $\uparrow$?

\end{itemize}

\end{frame}

%%%%%%%%%%%%%%%%%%%%%%%% new frame %%%%%%%%%%%%%%%%%%%%%%%%

\begin{frame}{Result 1: Rationality Extension}

\begin{figure}
\center
\includegraphics[width = 4.5 in]{figures/Figure1_CCEI_90}
%\caption{Computerized experiment in a classroom}
\end{figure}

\end{frame}

%%%%%%%%%%%%%%%%%%%%%%%% new frame %%%%%%%%%%%%%%%%%%%%%%%%

\begin{frame}
\frametitle{Result 1: Rationality Extension}

\begin{itemize}

\item Kolmogorov-Smirnov Test
\mk

\item Results:
\sk
	\begin{itemize}
	
	\item Dictator game: We find substantial increase of giving amount.
	\sk
	
	\item Public goods game: We do not find any significant increase of cooperation.
	
	\end{itemize}

\end{itemize}

\end{frame}

%%%%%%%%%%%%%%%%%%%%%%%% new frame %%%%%%%%%%%%%%%%%%%%%%%%

\begin{frame}{}

\vspace{0.05 in}
%\hspace{-0.15 in}
\centering \changefontsizes{8pt}
\begin{tabular}{l  lc lc lc lc}
\toprule
\multirow{2}{*}{\textbf{Collective CCEI}}	& & \multicolumn{7}{c}{\textbf{Coefficient}}  \\
		        \cline{3-9}
																& & Model 1 & & Model 2 & & Model 3 & & Model 4 \\
\hline
\hline
\multirow{2}{*}{CCEI\underline{{ }{ }}Max}  & &  $0.024^{***}$  & & $0.013$       & & $0.026^{**}$    & & $0.002$   \\
                                                 & & $(0.011)$         & & $(0.010)$    & & $(0.015)$          & &  $(0.014)$   \\
\multirow{2}{*}{CCEI\underline{{ }{ }}Max\underline{{ }{ }}Non-Mover} & & $0.049^{***}$ & & $0.044^{***}$  & &  $0.042^{***}$  & & $0.047^{***}$  \\
                                                & & $(0.012)$       & & $(0.011)$       & & $(0.014)$         & & $(0.015)$   \\
\multirow{2}{*}{CCEI\underline{{ }{ }}Distance}  & & $0.050^{***}$ & & $0.041^{***}$  & &  $0.036^{***}$  & & $0.041^{***}$  \\
                                     & & $(0.008)$       & & $(0.007)$        & & $(0.010)$          & & $(0.009)$   \\
\hline
\multirow{2}{*}{Risk\underline{{ }{ }}Aversion\underline{{ }{ }}Max}  & & $0.010$     & & $0.015^{**}$  & &  $0.014$   & & $0.015$     \\
                                              & & $(0.008)$   & & $(0.007)$      & & $(0.009)$  & & $(0.013)$   \\
\multirow{2}{*}{Risk\underline{{ }{ }}Aversion\underline{{ }{ }}Distance}  & & $-0.013$     & & $0.022$     & &  $0.015$   & & $0.028$  \\
                                   & & $(0.015)$   & & $(0.021)$   & & $(0.021)$  & & $(0.022)$   \\
\hline
\multirow{2}{*}{Non\underline{{ }{ }}Coed}  & & $0.004$      & & $0.003$     & &  $-0.005$  & & $0.012$  \\
                                             & & $(0.011)$   & & $(0.010)$   & &  $(0.010)$ & & $(0.010)$   \\
\multirow{2}{*}{Math\underline{{ }{ }}Score}  & &          & & $-0.007^{*}$  & &  $0.009^{*}$  & & $-0.005$  \\
                                             & &          & & $(0.004)$       & &  $(0.004)$     & & $(0.004)$ \\
\multirow{2}{*}{Math\underline{{ }{ }}Distance}  & &    & &  $-0.000$       & &  $-0.005$      & & $-0.005$  \\
                                                   & &          & & $(0.008)$       & &  $(0.008)$     & & $(0.009)$ \\
\multirow{2}{*}{Constant}      & & $0.308^{***}$   & & $0.184^{***}$    & & $0.237^{***}$ & & $0.127^{***}$  \\
                                             & &  $(0.047)$        & &  $(0.085)$         & &  $(0.051)$      & & $(0.092)$   \\
\hline
\hline
Class Fixed Effect                  & & Yes         & & Yes         & & Yes           & & Yes           \\
Individual Characteristics      & & Yes         & & Yes         & & Yes           & & Yes           \\
Friendship                              & & Yes         & & Yes         & & Yes           & & Yes           \\
Observations                  & & $9,377$ & & $9,377$ & & $4,584$   & & $4,793$  \\
$R$-squared                  & & $0.036$  & & $0.185$  & & $0.187$  & & $0.188$  \\
\bottomrule 
\end{tabular}

\end{frame}

%%%%%%%%%%%%%%%%%%%%%%%% new frame %%%%%%%%%%%%%%%%%%%%%%%%

\begin{frame}[noframenumbering]{}

\center
\textbf{Result 2: Preference Aggregation}

\end{frame}

%%%%%%%%%%%%%%%%%%%%%%%% new frame %%%%%%%%%%%%%%%%%%%%%%%%

\begin{frame}{Measurement: A Non-Parametric Measurement}

\begin{figure}
\center
\large 
\begin{tikzpicture}

% axes
\draw[->, thick] (0,0) -- (7,0) node[anchor = west] {};
\draw[->, thick] (0,0) -- (0,7) node[anchor = east] {};
		
% budget lines
\draw[thick] (6,0) -- (0,4);
\draw[thick] (0,6) -- (4,0);

% bundles
\draw (3.3, 6-3/2*3.3) node {\textbullet};
\draw (3.3, 6-3/2*3.3) node[anchor = north east] {$x'$};

\draw (6-3/2*3, 3) node {\textbullet};
\draw (6-3/2*3, 3) node[anchor = north east] {$x$};

% rationalization
\draw[red, ultra thick] (0,6-3/2*0.5) -- (4-0.5,0);

% texts
\small{
\draw (3, 7) node[anchor = west] {If there are two goods, a choice dataset satisfies the GARP};

\draw (3, 6.5) node[anchor = west] {if and only if it satisfies the WARP.};

\draw (3, 5.5) node[anchor = west] {$x'$ and $x$ violate GARP.};

\draw (3, 4.5) node[anchor = west] {If the budget line for $x'$ is deflated, then GARP is satisfied.};

\draw (3, 4) node[anchor = west] {Choose $e_v$ for each violation $v$.};

\draw (3, 3) node[anchor = west] {CCEI is defined as the supremum over all the numbers $e_v$'s.};

}

\end{tikzpicture}
\end{figure}

\end{frame}

%%%%%%%%%%%%%%%%%%%%%%%% new frame %%%%%%%%%%%%%%%%%%%%%%%%

\begin{frame}
\frametitle{Measurement}

\begin{itemize}

\item Expected utility form

\item Rank dependent utility form (RDU)

\item Aggregation of utility types

\item Assuming CARA, we can also estimate risk premium for each observation.

\end{itemize}

\end{frame}

%%%%%%%%%%%%%%%%%%%%%%%% new frame %%%%%%%%%%%%%%%%%%%%%%%%

\begin{frame}{Result 2: Risk Preference Aggregation}

\begin{figure}
\center
\includegraphics[width = 4.5 in]{figures/Figure2A_riskaversion}
%\caption{Computerized experiment in a classroom}
\end{figure}

\end{frame}

%%%%%%%%%%%%%%%%%%%%%%%% new frame %%%%%%%%%%%%%%%%%%%%%%%%

\begin{frame}{Result 2: Risk Preference Aggregation}

\begin{figure}
\center
\includegraphics[width = 4.5 in]{figures/Figure2B_riskpremium}
%\caption{Computerized experiment in a classroom}
\end{figure}

\end{frame}

%%%%%%%%%%%%%%%%%%%%%%%% new frame %%%%%%%%%%%%%%%%%%%%%%%%

\begin{frame}{}

\vspace{0.05 in}
%\hspace{-0.15 in}
\centering \changefontsizes{8pt}
\begin{tabular}{l  lc lc lc lc}
\toprule
\multirow{2}{*}{\textbf{Collective CCEI}}	& & \multicolumn{7}{c}{\textbf{Coefficient}}  \\
		        \cline{3-9}
																& & Model 1 & & Model 2 & & Model 3 & & Model 4 \\
\hline
\hline
\multirow{2}{*}{CCEI\underline{{ }{ }}Max}  & &  $0.024^{***}$  & & $0.013$       & & $0.026^{**}$    & & $0.002$   \\
                                                 & & $(0.011)$         & & $(0.010)$    & & $(0.015)$          & &  $(0.014)$   \\
\multirow{2}{*}{CCEI\underline{{ }{ }}Max\underline{{ }{ }}Non-Mover} & & $0.049^{***}$ & & $0.044^{***}$  & &  $0.042^{***}$  & & $0.047^{***}$  \\
                                                & & $(0.012)$       & & $(0.011)$       & & $(0.014)$         & & $(0.015)$   \\
\multirow{2}{*}{CCEI\underline{{ }{ }}Distance}  & & $0.050^{***}$ & & $0.041^{***}$  & &  $0.036^{***}$  & & $0.041^{***}$  \\
                                     & & $(0.008)$       & & $(0.007)$        & & $(0.010)$          & & $(0.009)$   \\
\hline
\multirow{2}{*}{Risk\underline{{ }{ }}Aversion\underline{{ }{ }}Max}  & & $0.010$     & & $0.015^{**}$  & &  $0.014$   & & $0.015$     \\
                                              & & $(0.008)$   & & $(0.007)$      & & $(0.009)$  & & $(0.013)$   \\
\multirow{2}{*}{Risk\underline{{ }{ }}Aversion\underline{{ }{ }}Distance}  & & $-0.013$     & & $0.022$     & &  $0.015$   & & $0.028$  \\
                                   & & $(0.015)$   & & $(0.021)$   & & $(0.021)$  & & $(0.022)$   \\
\hline
\multirow{2}{*}{Non\underline{{ }{ }}Coed}  & & $0.004$      & & $0.003$     & &  $-0.005$  & & $0.012$  \\
                                             & & $(0.011)$   & & $(0.010)$   & &  $(0.010)$ & & $(0.010)$   \\
\multirow{2}{*}{Math\underline{{ }{ }}Score}  & &          & & $-0.007^{*}$  & &  $0.009^{*}$  & & $-0.005$  \\
                                             & &          & & $(0.004)$       & &  $(0.004)$     & & $(0.004)$ \\
\multirow{2}{*}{Math\underline{{ }{ }}Distance}  & &    & &  $-0.000$       & &  $-0.005$      & & $-0.005$  \\
                                                   & &          & & $(0.008)$       & &  $(0.008)$     & & $(0.009)$ \\
\multirow{2}{*}{Constant}      & & $0.308^{***}$   & & $0.184^{***}$    & & $0.237^{***}$ & & $0.127^{***}$  \\
                                             & &  $(0.047)$        & &  $(0.085)$         & &  $(0.051)$      & & $(0.092)$   \\
\hline
\hline
Class Fixed Effect                  & & Yes         & & Yes         & & Yes           & & Yes           \\
Individual Characteristics      & & Yes         & & Yes         & & Yes           & & Yes           \\
Friendship                              & & Yes         & & Yes         & & Yes           & & Yes           \\
Observations                  & & $9,377$ & & $9,377$ & & $4,584$   & & $4,793$  \\
$R$-squared                  & & $0.036$  & & $0.185$  & & $0.187$  & & $0.188$  \\
\bottomrule 
\end{tabular}

\end{frame}

%%%%%%%%%%%%%%%%%%%%%%%% new frame %%%%%%%%%%%%%%%%%%%%%%%%

\begin{frame}[noframenumbering]{}

\center
\textbf{Result 3: Efficiency}

\end{frame}

%%%%%%%%%%%%%%%%%%%%%%%% new frame %%%%%%%%%%%%%%%%%%%%%%%%

\begin{frame}{Measurement}

\begin{itemize}

\item For given budget set, let $x_1^*$ and $x_2^*$ be the optimal portfolio choice of agent 1 and agent 2, respectively.
\mk

\item \textbf{Claim:} A group choice $x_c$ is \textbf{Pareto efficient} if and only if $x_c \in [x_1^*, x_2^*]$.
\mk

\item Given this, we measure the group inefficiency as the average utility loss:
	\sk
	\begin{align*}
	\text{Inefficiency}_{g} = \frac{1}{18} \sum_{k = 1}^{18} \frac{1}{2} \sum_{i = 1}^2 \frac{u_i(x) - u_i(x_c)}{u_i(x) - u_i(x_w)}.
	\end{align*}
	\sk
	where
	\sk
	\begin{itemize}
	
	\item $x_c$: group choice
	\sk
	
	\item $x_w$: the worst outcome in budget $k$
	
	\end{itemize}
	\mk
	
\item By definition, $\text{Loss}_{g} \in [0,1]$.

\end{itemize}

\end{frame}

%%%%%%%%%%%%%%%%%%%%%%%% new frame %%%%%%%%%%%%%%%%%%%%%%%%

\begin{frame}{Measurement: Distribution of Group Inefficiency}

\begin{figure}
\center
\includegraphics[width = 4.5 in]{figures/Figure3A_ULoss}
%\caption{Computerized experiment in a classroom}
\end{figure}

\end{frame}

%%%%%%%%%%%%%%%%%%%%%%%% new frame %%%%%%%%%%%%%%%%%%%%%%%%

\begin{frame}{Analysis}

\begin{itemize}

\item \textbf{Research Question:} How is the group inefficiency related to the group rationality and risk preference?

\end{itemize}

\end{frame}

%%%%%%%%%%%%%%%%%%%%%%%% new frame %%%%%%%%%%%%%%%%%%%%%%%%

\begin{frame}{Pareto Efficiency}

\begin{figure}
\center
\includegraphics[width = 4.5 in]{figures/Figure4A_ULoss_ccei}
%\caption{Computerized experiment in a classroom}
\end{figure}

\end{frame}

%%%%%%%%%%%%%%%%%%%%%%%% new frame %%%%%%%%%%%%%%%%%%%%%%%%

\begin{frame}{}

\vspace{0.05 in}
%\hspace{-0.15 in}
\centering \changefontsizes{8pt}
\begin{tabular}{l  lc lc lc lc}
\toprule
\multirow{2}{*}{\textbf{Collective CCEI}}	& & \multicolumn{7}{c}{\textbf{Coefficient}}  \\
		        \cline{3-9}
																& & Model 1 & & Model 2 & & Model 3 & & Model 4 \\
\hline
\hline
\multirow{2}{*}{CCEI\underline{{ }{ }}Max}  & &  $0.024^{***}$  & & $0.013$       & & $0.026^{**}$    & & $0.002$   \\
                                                 & & $(0.011)$         & & $(0.010)$    & & $(0.015)$          & &  $(0.014)$   \\
\multirow{2}{*}{CCEI\underline{{ }{ }}Max\underline{{ }{ }}Non-Mover} & & $0.049^{***}$ & & $0.044^{***}$  & &  $0.042^{***}$  & & $0.047^{***}$  \\
                                                & & $(0.012)$       & & $(0.011)$       & & $(0.014)$         & & $(0.015)$   \\
\multirow{2}{*}{CCEI\underline{{ }{ }}Distance}  & & $0.050^{***}$ & & $0.041^{***}$  & &  $0.036^{***}$  & & $0.041^{***}$  \\
                                     & & $(0.008)$       & & $(0.007)$        & & $(0.010)$          & & $(0.009)$   \\
\hline
\multirow{2}{*}{Risk\underline{{ }{ }}Aversion\underline{{ }{ }}Max}  & & $0.010$     & & $0.015^{**}$  & &  $0.014$   & & $0.015$     \\
                                              & & $(0.008)$   & & $(0.007)$      & & $(0.009)$  & & $(0.013)$   \\
\multirow{2}{*}{Risk\underline{{ }{ }}Aversion\underline{{ }{ }}Distance}  & & $-0.013$     & & $0.022$     & &  $0.015$   & & $0.028$  \\
                                   & & $(0.015)$   & & $(0.021)$   & & $(0.021)$  & & $(0.022)$   \\
\hline
\multirow{2}{*}{Non\underline{{ }{ }}Coed}  & & $0.004$      & & $0.003$     & &  $-0.005$  & & $0.012$  \\
                                             & & $(0.011)$   & & $(0.010)$   & &  $(0.010)$ & & $(0.010)$   \\
\multirow{2}{*}{Math\underline{{ }{ }}Score}  & &          & & $-0.007^{*}$  & &  $0.009^{*}$  & & $-0.005$  \\
                                             & &          & & $(0.004)$       & &  $(0.004)$     & & $(0.004)$ \\
\multirow{2}{*}{Math\underline{{ }{ }}Distance}  & &    & &  $-0.000$       & &  $-0.005$      & & $-0.005$  \\
                                                   & &          & & $(0.008)$       & &  $(0.008)$     & & $(0.009)$ \\
\multirow{2}{*}{Constant}      & & $0.308^{***}$   & & $0.184^{***}$    & & $0.237^{***}$ & & $0.127^{***}$  \\
                                             & &  $(0.047)$        & &  $(0.085)$         & &  $(0.051)$      & & $(0.092)$   \\
\hline
\hline
Class Fixed Effect                  & & Yes         & & Yes         & & Yes           & & Yes           \\
Individual Characteristics      & & Yes         & & Yes         & & Yes           & & Yes           \\
Friendship                              & & Yes         & & Yes         & & Yes           & & Yes           \\
Observations                  & & $9,377$ & & $9,377$ & & $4,584$   & & $4,793$  \\
$R$-squared                  & & $0.036$  & & $0.185$  & & $0.187$  & & $0.188$  \\
\bottomrule 
\end{tabular}

\end{frame}

%%%%%%%%%%%%%%%%%%%%%%%% new frame %%%%%%%%%%%%%%%%%%%%%%%%

\begin{frame}{Conclusion}

\begin{itemize}

\item 

\end{itemize}

\end{frame}

\appendix

%%%%%%%%%%%%%%%%%%%%%%%% new frame %%%%%%%%%%%%%%%%%%%%%%%%

\begin{frame}{}

\center
\textbf{Robustness}

\end{frame}

%%%%%%%%%%%%%%%%%%%%%%%% new frame %%%%%%%%%%%%%%%%%%%%%%%%

\begin{frame}{Varian's Efficiency Index}

\begin{itemize}

\item 

\end{itemize}

\end{frame}

%%%%%%%%%%%%%%%%%%%%%%%% new frame %%%%%%%%%%%%%%%%%%%%%%%%

\begin{frame}{Introduction: Risk Perception}

\begin{itemize}

\item[1.] Genetics: White males wre more likely to perceive risks as being smaller (Bickerstaff, 2004; Flynn et al., 1994).
\mk

\item[2.] Psychology: 
\sk
	\begin{itemize}
	
	\item availability bias 
	\sk
	
	\item experience
	\end{itemize}
	\mk
	
\item[3.] Knowledge and Information: public v.s. experts
\mk
	
\item[4.] Other factors
\sk
	\begin{itemize}
	
	\item culture: Chinese individuals are significantly less risk-averse than individuals from Western countries when making financial decisions (Weber and Hsee, 1998).
	\sk
	
	\item power: high-power groups adopt a more positive attitude toward potential risks (Anderson and Galinsky, 2006; Magee et al., 2007; Geng et al, 2018).
	\sk
	
	\item social inequality: Sweden people with foreign backgrounds did perceive risks as higher than native people, but no difference between men and women.
	
	\end{itemize}

\end{itemize}

\end{frame}

%%%%%%%%%%%%%%%%%%%%%%%% new frame %%%%%%%%%%%%%%%%%%%%%%%%

\begin{frame}{Introduction: Implication}

\begin{itemize}

\item Health
	\sk
	\begin{itemize}
	
	\item ``미세먼지의 위험성에 대해서 잘 인식하지 못하는 사람들도 적지 않았다. 경기도 고양시에서 폐지 줍는 일을 하는 조모(82) 씨는 미세먼지가 무섭지 않다고 했다. 그는 노인들이 미세먼지에 더 취약하다는 지적에 “\textcolor{tred}{70평생 먼지 들이켜고도 잘 살았다}”고 말했다'' \\ \ \\
	-- 헤럴드 경제, \textit{[미세먼지의 습격⑦] 거리가 일터인 택배원ㆍ배달원 “마스크쓰고 일 못해요”, 2018. 1. 17.} --
	
	\end{itemize}

\end{itemize}

\end{frame}

%%%%%%%%%%%%%%%%%%%%%%%% new frame %%%%%%%%%%%%%%%%%%%%%%%%

\begin{frame}{Introduction: Implication}

\begin{itemize}

\item Health
	\sk
	\begin{itemize}
	
	\item 비용-편익 분석의 핵심은 특정 정책의 비용과 편익에 대한 정확한 추정을 기반으로 한다. 결과적으로 효과적이고 합리적인 정책은 그 정책 대상에 대한 정확한 가치의 추정 을 바탕으로 그에 상응하는 정책을 수립하는 데에서 출발할 수 있을 것이다. 환경정책도 여기서 예외가 될 수는 없다. 즉, 효과적이고 합리적인 환경정책은 해당 환경자원에 대한 정확한 가치의 추정이 선행되어야 한다는 것이고 이렇게 추정된 가치에 대한 자료를 근거로 정책의 우선순위나 당위성을 확보할 수 있다는 것이다. 그런데, 환경정책에서 가장 어려운 부분이 바로 여기에 존재한다. 환경자원은 그 특성상 시장이 존재하지 않는 경우가 많으며 기타 시장재화와는 다르게 특정 환경자원의 화폐적 가치를 직접적으로 추정하기가 어렵다. 환경자원의 가치 추정에서의 어려움을 극복하기 위해 제안된 가치평 가기법들은 여러 가지가 존재한다. 이 중에서도 가장 광범위하게 사용되고 있는 기법은 가상 시장에 기반을 두어 그 환경자원의 가치를 직접적으로 유도하는 방식인 진술선호법 중 하나인 조건부가치평가법(contingent valuation method)을 꼽을 수 있다.
	
	\end{itemize}
	\mk
	
\item Policy: 시민참여단의 성, 연령, 거주 권역, 교육수준 등의 특성에 따라 건설 재개 및 중단에 대한 응답에 차이가 있는지를 알아보기 위해 차이검정을 실시했다.
먼저 성별에 따른 차이를 보면 남자 66.3%, 여자 52.7%가 건설 재개를 선택했다. 남성과 여성 모두 과반이 건설 재개를 지지했으나, 남성이 여성에 비해 건설 재개를 지지하는 비율이 13.6%p 높았으며 성별에 따른 차이는 통계적으로 유의미했다.
연령별로 보면 20대는 56.8%, 30대는 52.3%, 40대는 45.3%, 50대는 60.5%, 60대 이상은 77.5% 가 건설 재개에 응답했다. 40대를 제외한 연령층 모두에서 건설 재개를 지지하는 비율이 높았고, 특히 60대 이상 노년층에서 건설 재개에 동의하는 비율이 가장 높았다.
권역별로 살펴보면 수도권 시민참여단은 건설 재개와 중단에 대해 전국 평균과 유사한 경향을 보여 주었다. 호남권 시민참여단은 건설 중단을, 중부권과 영남권 시민참여단은 건설 재개를 더 많이 지지한 것으로 나타났다. 하지만 권역에 따른 건설 재개/중단에 대한 입장 차이가 통계적으로 유의미하지는 않았다.
교육수준별로 살펴보면 학력에 관계없이 건설 재개를 지지하는 입장이 과반인 가운데, 고졸이하 학력 집단이 대학재학 이상 집단보다 건설 재개를 지지하는 비율이 17.7%p 더 높았다.
                    
\mk

\item Agricultural activities: Farmers tend to be risk averse and that risk aversion determines their diversification strategies (Lin et al., 1974; Fafchamps, 1992).

\end{itemize}

\end{frame}

%%%%%%%%%%%%%%%%%%%%%%%% new frame %%%%%%%%%%%%%%%%%%%%%%%%

%\begin{frame}{Introduction to Experimental Economics}

%\center
%\includegraphics[height = 3 in]<1>{pfigs/dan}
%\includegraphics[height = 3 in]<2>{pfigs/roth}
%\includegraphics[height = 3 in]<3>{pfigs/thaler1}
%\includegraphics[height = 3 in]<4>{pfigs/thaler2}

%\end{frame}

\end{document}


%%%%%% Inventory

%\center
%\includegraphics[width = 1.8 in]{textbook}
