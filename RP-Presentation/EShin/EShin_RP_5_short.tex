\documentclass[10pt, xcolor={dvipsnames,table}]{beamer}
%\documentclass[10pt,handout]{beamer} % for handout or editing

\usetheme{Madrid}
\usecolortheme{dolphin} % Simple and clean template
\setbeamertemplate{footline}[frame number]
\usefonttheme{professionalfonts}

% reduce the margins
\setbeamersize{text margin left = 0.25 in}
\setbeamersize{text margin right = 0.25 in}

%
\usepackage{graphicx}
\usepackage{amssymb}
\usepackage{amsfonts}
\usepackage{amsmath}
\usepackage{amsthm}
\usepackage{mathrsfs}
\usepackage{color}
\usepackage{setspace}
\usepackage{bbm}
\usepackage{subfigure}

%without auto-caption labeling
\usepackage{stackengine}

%%%%% for graphics in latex (e.g. arrows)
\usepackage{tikz}
\usetikzlibrary{arrows,chains,matrix,positioning,scopes,automata}
\tikzset{
  invisible/.style={opacity=0},
  visible on/.style={alt={#1{}{invisible}}},
  alt/.code args={<#1>#2#3}{%
    \alt<#1>{\pgfkeysalso{#2}}{\pgfkeysalso{#3}} % \pgfkeysalso doesn't change the path
  },
}

%%%%% draw functions by using tikz
\usepackage{pgfplots}

%colors
\definecolor{tred}{rgb}{0.35, 0.0, 0.0}
\definecolor{tblue}{rgb}{0.0, 0.0, 0.35}
\definecolor{grey}{rgb}{0.4, 0.4, 0.4}
\definecolor{wbgrey}{rgb}{0.9, 0.9, 1}
\definecolor{wrgrey}{rgb}{1, 0.9, 0.9}
\definecolor{wggrey}{rgb}{0.9, 1, 0.9}
\definecolor{periwinkle}{rgb}{0.8, 0.8, 1.0}
\definecolor{purpleheart}{rgb}{0.41, 0.21, 0.61}

%block color
\setbeamercolor{block title}{bg=purpleheart!90, fg=white}

\usepackage{mathtools}
\usepackage[export]{adjustbox}

% set separated appendix numbers
\usepackage{appendixnumberbeamer}

%%% vertical skips
\newcommand{\sk}{\vspace{0.1 in}}
\newcommand{\mk}{\vspace{0.2 in}}
\newcommand{\bk}{\vspace{0.3 in}}
\newcommand{\nk}{\vspace{- 0.1 in}}


\DeclarePairedDelimiter{\ceil}{\lceil}{\rceil}

\newcommand{\scrG}{\mathscr{G}}
\newcommand{\scrF}{\mathscr{F}}

\newtheorem{proposition}[theorem]{Proposition}

% turn off navigation symbols
\setbeamertemplate{navigation symbols}{}

% for appendix
\usepackage{appendixnumberbeamer}

%\usepackage{bm}         % For typesetting bold math (not \mathbold)
%\logo{\includegraphics[height=0.6cm]{caltechslide.pdf}}

%%%%%%%%%%%%%%%%%%%%%%%%%%%%%%%% Here are some custumized commands %%%%%%%%%%%%%%%%%%%%%%%%%%%%%%%%%%%%%%

\newtheorem{axiom}{Axiom}

\newcommand{\plim}{\overset{p}{\longrightarrow}}% converges in probability to.
\newcommand{\rthlim}{\overset{L^r}{\longrightarrow}}% converges in r-th mean
\newcommand{\ip}[1]{\langle #1 \rangle}
\newcommand{\Lim}[1]{\raisebox{0.5ex}{\scalebox{0.8}{$\displaystyle \lim_{#1}\;$}}}

\newcommand{\argmax}{\operatornamewithlimits{argmax}}
\newcommand{\argmin}{\operatornamewithlimits{argmin}}
\newcommand{\bigtimes}{\operatornamewithlimits{\BIGOP{\times}}}

\newcommand{\pr}{\mathbbm{P}}
\newcommand{\ex}{\mathbbm{E}}
\newcommand{\norm}[1]{\left|\left|#1\right|\right|}

% Number systems
\newcommand{\NN}{\mathbb{N}}
\newcommand{\ZZ}{\mathbb{Z}}
\newcommand{\QQ}{\mathbb{Q}}
\newcommand{\RR}{\mathbb{R}}

\newcommand{\GG}{\bm{G}}
\newcommand{\RN}{\bm{N}}
\newcommand{\RM}{\bm{M}}
\newcommand{\RZ}{\bm{Z}}
\newcommand{\bI}{\bm{I}}
\newcommand{\XX}{\bm{X}}

\newcommand{\ff}{\bm{f}}
\newcommand{\xx}{\bm{x}}
\newcommand{\ba}{\bm{a}}
\newcommand{\rz}{\bm{z}}


\newcommand{\idc}{\mathbbm{1}}

\newcommand{\calE}{\mathcal{E}}
\newcommand{\calP}{\mathcal{P}}
\newcommand{\calR}{\mathcal{R}}
\newcommand{\calF}{\mathcal{F}}
\newcommand{\calV}{\mathcal{V}}
\newcommand{\calI}{\mathcal{I}}

\newcommand{\rmu}{\bm{\mu}}
\newcommand{\Rm}{\bm{m}}
\newcommand{\pp}{\bm{p}}
\newcommand{\dd}{\bm{d}}

\newcommand{\DD}{\mathcal{D}}

\definecolor{tred}{RGB}{200,50,80}
\definecolor{tgreen}{RGB}{80,180,80}
\definecolor{tblue}{RGB}{50,50,250}
\definecolor{tpupple}{RGB}{150,50,200}

% reduce space around align
\usepackage{etoolbox}
\newcommand{\zerodisplayskips}{%
  \setlength{\abovedisplayskip}{0.05 in}
  \setlength{\belowdisplayskip}{0.05 in}
  \setlength{\abovedisplayshortskip}{0.05 in}
  \setlength{\belowdisplayshortskip}{0.05 in}}
\appto{\normalsize}{\zerodisplayskips}
\appto{\small}{\zerodisplayskips}
\appto{\footnotesize}{\zerodisplayskips}

\setbeamerfont{caption}{size=\scriptsize}
\setbeamerfont{footnote}{size=\scriptsize}


% include pdf version 1.6
%\pdfoptionpdfminorversion=7 


\usepackage{kotex}
\usepackage{textcomp} % this is to use Korean currency, Won.

\usepackage{scrextend}

\usepackage{fancybox}     
\usepackage{booktabs}              
\usepackage{multirow}

%\newcolumntype{?}{!{\vrule width 1pt}}

\usetikzlibrary{decorations.pathreplacing,angles,quotes} % to use curled braces
\usepackage{changepage} % to move page

%%%% testing environments

%\usepackage{thmtools}
%\usepackage{boiboites} % this requires boiboites.sty in the folder
\usepackage{eshinbox}

\newcounter{thm}
\numberwithin{thm}{section}

\newboxedtheorem[boxcolor=black, background=gray!5, titlebackground=gray!20,
titleboxcolor = black]{thm}{Theorem}{thm}

\newboxedtheorem[boxcolor=black, background=gray!5, titlebackground=gray!20,
titleboxcolor = black]{prp}{Proposition}{thm}

%%%%%%%%%%%%%%%%%%%%%%%%%%%%%%%% START %%%%%%%%%%%%%%%%%%%%%%%%%%%%%%%%%%%%%%

\title[Short title of the talk]{Rationality and Preference Aggregation of Group Decision under Risk}
\author{Syngjoo Choi (SNU), Booyuel Kim (KDIS), Minseon Park (Wisconsin), Yoonsoo Park (KDI), Euncheol Shin (KHU)}
\institute[]
{Presenter: Euncheol Shin (Kyung Hee University) \\
\medskip
{\tt{eshin.econ@khu.ac.kr}}
}
\date{%June, 2017
\today %will show current date.
%324 Alternatively, you can specify a date.
}

%
\begin{document}
%
\begin{frame}
\titlepage
\end{frame}

%%%%%%%%%%%%%%%%%%%%%%%% new frame %%%%%%%%%%%%%%%%%%%%%%%%

\begin{frame}{Introduction}

\begin{itemize}

\item In various contexts, many important decisions are made by groups.
\mk

\item Individual heterogeneity exists in various dimensions:
\sk
	\begin{itemize}
	
	\item Risk preference: risk assessment in environmental policy
	\sk
	
	\item Time preference: household savings and consumption decisions
	\sk

	\item Rationality
	
	\end{itemize}
	\mk
	
\item It is important to understand how individual heterogeneity in a collective influences final outcomes.

\end{itemize}

\end{frame}

%%%%%%%%%%%%%%%%%%%%%%%% new frame %%%%%%%%%%%%%%%%%%%%%%%%

\begin{frame}{Introduction: Research Questions}

\begin{itemize}

\item[1.] Rationality extension:
	\sk
	\begin{itemize}
	
	\item Do rational members make more collectively rational decisions?
	
	\end{itemize}
	\mk

\item[2] Risk preference aggregation:
	\sk
	\begin{itemize}
	
	\item Are individual's risk preferences reflected into that of a group?
	
	\end{itemize}
	\mk
	
\item[3.] Efficiency and welfare:
	\sk
	\begin{itemize}
	
	\item How is the efficiency of group decisions related to individual's rationality and preferences?
	\sk
	
	\item How is social welfare related to individual' rationality and preferences?
	
	\end{itemize}
	
\end{itemize}

\end{frame}

%%%%%%%%%%%%%%%%%%%%%%%% new frame %%%%%%%%%%%%%%%%%%%%%%%%

\begin{frame}[noframenumbering]{}

\center
\textbf{Experimental Design}

\end{frame}

%%%%%%%%%%%%%%%%%%%%%%%% new frame %%%%%%%%%%%%%%%%%%%%%%%%

\begin{frame}{Screenshot}

\begin{figure}
\center
\includegraphics[width = 3.8 in]{figures/expshot}
\end{figure}

\end{frame}

%%%%%%%%%%%%%%%%%%%%%%%% new frame %%%%%%%%%%%%%%%%%%%%%%%%

\begin{frame}{Experimental Design (Choi et al.,2007; Choi et al., 2014)}
\nk

\begin{figure}
\center
\large 
\begin{tikzpicture}

% axes
\draw[->, thick] (0,0) -- (7,0) node[anchor = west] {$x_r$};
\draw[->, thick] (0,0) -- (0,7) node[anchor = east] {$x_b$};
		
% budget lines
\draw[thick] (6,0) -- (0,3);

% 45 degree line
\draw[dotted, thick] (0,0) -- (4,4) node[anchor = south west] {$x_r = x_b$};
\draw (0.8, 0) node[anchor = south] {$45^\circ$};

% prices
\draw[->, thick] (6,0) -- (6+1/2, 1) node[anchor = west] {$p = (p_r, p_b)$};

% bundles
\draw (2, 2) node {\textbullet};
\draw (2, 2.1) node[anchor = south] {$x$};

\draw (3, 6-3/2*3) node {\textbullet};
\draw (3, 6-3/2*3) node[anchor = north east] {$x'$};

\draw (6, 0) node {\textbullet};
\draw (6, - 0.6) node[anchor = south] {$x''$};

% texts
\visible<1>{
\small{
\draw (3, 7) node[anchor = west] {Two \textbf{equally likely} states: $\textcolor{tred}{R}$ and $\textcolor{tblue}{B}$.};

\draw (3, 6.5) node[anchor = west] {There are two associated Arrow securities.};

\draw (3, 6) node[anchor = west] {$x_r$ is the demand for the security that pays off in state $\textcolor{tred}{R}$.};

\draw (3, 5.5) node[anchor = west] {$x_b$ is the demand for the security that pays off in state $\textcolor{tblue}{B}$.};

\draw (3, 5) node[anchor = west] {Budget constraint: $p_r x_r + p_b x_b = 1$.};
}
}

\visible<2>{
\small{
\draw (3, 7) node[anchor = west] {In this example, $p_b > p_r$.};

\draw (3, 6.5) node[anchor = west] {Risk neutral agent will choose $x''$.};

\draw (3, 6) node[anchor = west] {Extremely risk averse agent will choose $x$.};

\draw (3, 5.5) node[anchor = west] {Intermediately risk averse agent will choose $x'$.};

\draw (3, 5) node[anchor = west] {risk aversion (RA) = $\frac{x_{\text{expensive}} }{ x_{\text{expensive}} + x_{\text{cheap}} }$};

%\draw (3, 5) node[anchor = west] {$r = \frac{x_r}{x_r + x_b}$};
}
}

\end{tikzpicture}
\end{figure}

\end{frame}

%%%%%%%%%%%%%%%%%%%%%%%% new frame %%%%%%%%%%%%%%%%%%%%%%%%

\begin{frame}{Measurement: Afriat's Efficiency Index (a.k.a. CCEI)}

\begin{adjustwidth}{-0.5cm}{}
\begin{figure}
\center
\large 
\begin{tikzpicture}

% axes
\draw[->, thick] (0,0) -- (7,0) node[anchor = west] {};
\draw[->, thick] (0,0) -- (0,7) node[anchor = east] {};
		
% budget lines
\draw[thick] (6,0) -- (0,4);
\draw[thick] (0,6) -- (4,0);

% bundles
\draw (3.5, 6-3/2*3.5) node {\textbullet};
\draw (3.5, 6-3/2*3.5) node[anchor = north east] {$x'$};

\draw (6-3/2*3, 3) node {\textbullet};
\draw (6-3/2*3, 3) node[anchor = north east] {$x$};

% rationalization
\visible<2,4>{
\draw[tred, ultra thick, dashed] (0,6-3/2*0.5) -- (4-0.5,0);
%\draw[tblue, ultra thick, dashed] (6-3/2*0.5, 0) -- (0, 4-0.5);
}

\visible<2>{
\draw[decoration={brace, raise=0pt, amplitude=12pt}, xshift=0pt,yshift=0pt, decorate, thick]
  (0,0) -- node[left=1pt,midway,xshift=-0.3cm] {$e_v$} (0,6-3/4);
}

\visible<3->{
\draw[tblue, ultra thick, dashed] (6-3/2*0.88, 0) -- (0, 4-0.88);
}

% texts
\small{
\draw (3, 7) node[anchor = west] {If there are two goods, a choice dataset satisfies the GARP};

\draw (3, 6.5) node[anchor = west] {if and only if it satisfies the WARP.};

\draw (3, 5.5) node[anchor = west] {$x'$ and $x$ violate the GARP.};

\draw (3, 4.5) node[anchor = west] {If the budget line for $x'$ is deflated, the GARP is satisfied.};

\draw (3, 4) node[anchor = west] {Choose $e_v$ for each violation $v$.};

\draw (3, 3) node[anchor = west] {\textbf{Critical cost efficiency index} (CCEI) is defined as the};

\draw (3, 2.5) node[anchor = west] {\textbf{supremum} over all the numbers $e_v$'s.};
}

\end{tikzpicture}
\end{figure}
\end{adjustwidth}

\end{frame}

%%%%%%%%%%%%%%%%%%%%%%%% new frame %%%%%%%%%%%%%%%%%%%%%%%%

\begin{frame}{Field}

\begin{figure}
\center
\includegraphics[width = 4.0 in]{figures/pblrealshot}
\end{figure}

\end{frame}

%%%%%%%%%%%%%%%%%%%%%%%% new frame %%%%%%%%%%%%%%%%%%%%%%%%

\begin{frame}{Procedure and Subjects}

\begin{itemize}

\item We conducted the experiment in 12 middle schools in Daegu.
\mk

\item The number of students: 1572.
\mk

\item The number of groups: 786.
\mk

\item The instructions were read by an experimenter in each classroom.
\mk
	
\item Each subject participated in two sessions: individual and group decisions.

\end{itemize}

\end{frame}

%%%%%%%%%%%%%%%%%%%%%%%% new frame %%%%%%%%%%%%%%%%%%%%%%%%

\begin{frame}[noframenumbering]{Example of Choice Data}

\begin{figure}
\center
%\includegraphics[width = 4.5 in]<1>{figures/Figure_Choice_Example1}
\includegraphics[width = 4.5 in]{figures/Figure_Choice_Example2}
\end{figure}

\end{frame}

%%%%%%%%%%%%%%%%%%%%%%%% new frame %%%%%%%%%%%%%%%%%%%%%%%%

\begin{frame}{Rationality Extension: Research Question}

\center
Individual Rationality $\uparrow$ $\Rightarrow$ Collective Rationality $\uparrow$?

\end{frame}

%%%%%%%%%%%%%%%%%%%%%%%% new frame %%%%%%%%%%%%%%%%%%%%%%%%

\begin{frame}{Rationality Extension: First-Order Stochastic Dominance}

\begin{figure}
\center
\includegraphics[width = 4.5 in]{figures/Figure1_CCEI_90}
%\caption{Computerized experiment in a classroom}
\end{figure}

\end{frame}

%%%%%%%%%%%%%%%%%%%%%%%% new frame %%%%%%%%%%%%%%%%%%%%%%%%

\begin{frame}{Risk Preference Aggregation: Research Question}

\center
Individual risk aversion $\uparrow$ $\Rightarrow$ Collective risk aversion $\uparrow$?

\end{frame}

%%%%%%%%%%%%%%%%%%%%%%%% new frame %%%%%%%%%%%%%%%%%%%%%%%%

\begin{frame}{Preference Aggregation: FOSD by Relative Ratio}

\begin{figure}
\center
\includegraphics[width = 4.5 in]{figures/Figure2A_riskaversion}
%\caption{Computerized experiment in a classroom}
\end{figure}

\end{frame}

%%%%%%%%%%%%%%%%%%%%%%%% new frame %%%%%%%%%%%%%%%%%%%%%%%%

\begin{frame}[noframenumbering]{}

\center
\textbf{Result 3: Efficiency and Welfare}

\end{frame}

%%%%%%%%%%%%%%%%%%%%%%%% new frame %%%%%%%%%%%%%%%%%%%%%%%%

\begin{frame}{Measurement: Idea}

\begin{itemize}

\item We analyze the \textbf{quality} of collective decisions as a function of the degrees of rationality and preference alignment.
\mk

\item Idea: 
\sk
	\begin{itemize}
	
	\item We consider a class of utility functions over lotteries.
	\sk
	
	\item For each subject, we estimate the utility function parametrically.
	\sk
	
	\item We characterize a set of Pareto efficient choices.
	\sk
	
	\item For collective choices which are \textbf{not} Pareto efficient, we measure the degree of welfare loss.
	
	\end{itemize}

\end{itemize}

\end{frame}

%%%%%%%%%%%%%%%%%%%%%%%% new frame %%%%%%%%%%%%%%%%%%%%%%%%

\begin{frame}{Measurement: Utility Estimation}

\begin{itemize}

\item We restrict our attention to a CARA utility function over outcomes.
\mk

\item We consider two different types of utility function over lotteries:
	\sk
	\begin{itemize}
	\item Expected utility (EU)
	\sk
	
	\item Disappointment aversion utility (DAU).
	\end{itemize}
	\mk

\item We estimate $\rho$ and $\beta$ simultaneously by using a combination of a bootstrapping and the non-linear least square (NLLS) methods:
	\sk
	\begin{itemize}
	
	\item[1] Find subsample of size 18 with replacement.
	\sk
	
	\item[2] For given subsample, estimate $\alpha$ and $\rho$  by NLLS.
	\sk
	
	\item[3] Repeat the above for 250 times.
	\sk
	
	\item[4] If $0.5 \in [\alpha_{2.5}, \alpha_{97.5}]$, then set $\alpha = 0.5$ as an EU.
	\sk
	
	\item[4'] Otherwise, set $\alpha = \overline{\alpha}$ as a DAU.
	
	\end{itemize}

\end{itemize}

\end{frame}

%%%%%%%%%%%%%%%%%%%%%%%% new frame %%%%%%%%%%%%%%%%%%%%%%%%

\begin{frame}{Measurement: Efficiency and Welfare Loss}
\nk

\begin{figure}
\center
\large 
\begin{tikzpicture}

% axes
\draw[->, thick] (0,0) -- (7,0) node[anchor = west] {$x_r$};
\draw[->, thick] (0,0) -- (0,7) node[anchor = east] {$x_b$};
		
% budget lines
\draw[thick] (6,0) -- (0,3);

\visible<1->{
% indifference curve1
\draw[thick, tred] (2.2, 1.9) to [out=-20, in=180] (6,1) node[anchor = south west] {};
\draw[thick, tred] (2.2, 1.9) to [out=150, in=-90] (0.2,6) node[anchor = south west] {};
% choice
\draw (2.2, 1.9) node {\textbullet};
\draw (2.2, 1.9) node[anchor = south west] {$x^*$};
}

\visible<1->{
% indifference curve2
\draw[thick, tblue] (5, 0.48) to [out=-20, in=170] (6,0.2) node[anchor = south west] {};
\draw[thick, tblue] (5, 0.48) to [out=140, in=-80] (1.5,5) node[anchor = south west] {};
% choice
\draw (5, 0.48) node {\textbullet};
\draw (5, 0.48) node[anchor = south west] {$y^*$};
}

\visible<2->{
\small
% budget lines
\draw[ultra thick] (2.2,1.9) -- (5,0.48);
\draw (5, 6) node[anchor = south] {The set of Pareto efficient choices};
\draw[thick, ->] (5,6) -- (3.5,1.3);
}

\end{tikzpicture}
\end{figure}

\end{frame}

%%%%%%%%%%%%%%%%%%%%%%%% new frame %%%%%%%%%%%%%%%%%%%%%%%%

\begin{frame}{Measurement: Efficiency and Welfare Loss}
\nk

\begin{figure}
\center
\large 
\begin{tikzpicture}

% axes
\draw[->, thick] (0,0) -- (7,0) node[anchor = west] {$U_1$};
\draw[->, thick] (0,0) -- (0,7) node[anchor = east] {$U_2$};

\visible<1->{
% feasible utilities
\draw[thick] (0, 0) to [out=2, in=-90] (6,2) to [out=90, in=-45] (4.8,4.8);
\draw[thick] (0, 0) to [out=88, in=180] (2,6) to [out=0, in=135] (4.8,4.8);
\draw[ultra thick] (6,2) to [out=90, in=-45] (4.8,4.8);
\draw[ultra thick] (2,6) to [out=0, in=135] (4.8,4.8);

% guild lines
\draw[dotted, thick] (6,0) -- (6,7) node[anchor = south west] {};
\draw (6, 0) node[anchor = north] {$U_1^{\max}$};
%\draw[dotted, thick] (2,0) -- (2,7) node[anchor = south west] {};
%\draw (0, 0) node[anchor = north] {$U_1^{min}$};

\draw[dotted, thick] (0,6) -- (7,6) node[anchor = south west] {};
\draw (0, 6) node[anchor = east] {$U_2^{\max}$};
%\draw[dotted, thick] (0,2) -- (7,2) node[anchor = south west] {};
\draw (0, 0) node[anchor = north east] {$(0,0)$};
}

\visible<2->{
% choice
\draw (5, 0.38) node {\textbullet};
\draw[dotted, thick] (5,0) -- (5,7) node[anchor = south west] {};
\draw[dotted, thick] (0,0.38) -- (7,0.38) node[anchor = south west] {};
\draw (5, 0.38) node[anchor = south east] {$(U_1^o, U_2^o)$};
}


\visible<3->{
\draw[decoration={brace, raise=0pt, amplitude=8pt, mirror}, xshift=0pt,yshift=0pt, decorate, ultra thick, tblue]
(0,0) -- node[below=1pt,midway,yshift=-0.1cm] {$(a)$} (6,0);
\draw[decoration={brace, raise=0pt, amplitude=8pt}, xshift=0pt,yshift=0pt, decorate, ultra thick, tblue]
(5,0.38) -- node[above=1pt,midway,yshift=0.1cm] {$(b)$} (6,0.38);
}

\visible<4->{
\draw[decoration={brace, raise=0pt, amplitude=8pt}, xshift=0pt,yshift=0pt, decorate, ultra thick, tred]
(0,0) -- node[left=1pt,midway,xshift=-0.1cm] {$(c)$} (0,6);
\draw[decoration={brace, raise=0pt, amplitude=8pt, mirror}, xshift=0pt,yshift=0pt, decorate, ultra thick, tred]
(5,0.38) -- node[right=1pt,midway,xshift=0.1cm] {$(d)$} (5,6);
}


\end{tikzpicture}
\end{figure}

\end{frame}

%%%%%%%%%%%%%%%%%%%%%%%% new frame %%%%%%%%%%%%%%%%%%%%%%%%

\begin{frame}{Efficiency and Welfare: Measurement}

\begin{itemize}

\item We focus on the group choices which are not Pareto efficient (60\%).
\mk

\item For those choices, we measure \textbf{welfare loss} of a group as
	\sk
	\begin{align*}
	\text{Welfare Loss} = \frac{1}{18} \sum_{k = 1}^{18} \frac{1}{2} \sum_{i = 1}^2 \frac{U_i(x_{ikb}) - U_i(x_{ikc})}{U_i(x_{ikb}) - U_i(x_{ikw})},
	\end{align*}
	\sk
	where
	\sk
	\begin{itemize}
	
	\item $x_{ikc}$: group choice in $k$-th round
	\sk
	
	\item $x_{ikb}$: member $i$'s best choice in $k$-th round
	\sk
	
	\item $x_{ikw}$: member $i$'s worst choice in $k$-th round.
	
	\end{itemize}
	\mk
	
\item By definition, $\text{Welfare Loss} \in [0,1]$.

\end{itemize}

\end{frame}

%%%%%%%%%%%%%%%%%%%%%%%% new frame %%%%%%%%%%%%%%%%%%%%%%%%

\begin{frame}{Efficiency and Welfare: Distribution of Welfare Loss}

\begin{figure}
\center
\includegraphics[width = 4.5 in]{figures/Figure3A_ULoss}
%\caption{Computerized experiment in a classroom}
\end{figure}

\end{frame}

%%%%%%%%%%%%%%%%%%%%%%%% new frame %%%%%%%%%%%%%%%%%%%%%%%%

\begin{frame}{Welfare: FOSD by Group Rationality}

\begin{figure}
\center
\includegraphics[width = 4.5 in]{figures/Figure4A_ULoss_ccei}
%\caption{Computerized experiment in a classroom}
\end{figure}

\end{frame}

%%%%%%%%%%%%%%%%%%%%%%%% new frame %%%%%%%%%%%%%%%%%%%%%%%%

\begin{frame}{Conclusion}

\begin{itemize}

\item We measure rationality and risk preference in individual and group levels.
\mk

\item We observe rationality extension and preference aggregation.
\mk

\item We develop a measure of efficiency and utility loss of group decisions.
\mk

\item We find that rational groups are more likely to make efficient decisions.
	\mk
	
\item Our main findings are robust with respect to
	\sk
	\begin{itemize}
	\item another rationality measure (Varian's efficiency index)
	\sk
	
	\item other cutoff values of CCEI (0.99 or 0.95)
	\sk
	
	\item another measure of risk preferences (risk premium).
	
	\end{itemize}
	
\end{itemize}

\end{frame}

%%%%%%%%%%%%%%%%%%

\end{document}


%%%%%% Inventory

%\center
%\includegraphics[width = 1.8 in]{textbook}
