\documentclass[10pt, xcolor={dvipsnames,table}]{beamer}
%\documentclass[11pt,handout]{beamer} % for handout or editing

\usetheme{Madrid}
\usecolortheme{dolphin} % Simple and clean template
\setbeamertemplate{footline}[frame number]
\usefonttheme{professionalfonts}

% reduce the margins
\setbeamersize{text margin left = 0.25 in}
\setbeamersize{text margin right = 0.25 in}

%
\usepackage{graphicx}
\usepackage{amssymb}
\usepackage{amsfonts}
\usepackage{amsmath}
\usepackage{amsthm}
\usepackage{mathrsfs}
\usepackage{color}
\usepackage{setspace}
\usepackage{bbm}
\usepackage{subfigure}

%without auto-caption labeling
\usepackage{stackengine}

%%%%% for graphics in latex (e.g. arrows)
\usepackage{tikz}
\usetikzlibrary{arrows,chains,matrix,positioning,scopes,automata}
\tikzset{
  invisible/.style={opacity=0},
  visible on/.style={alt={#1{}{invisible}}},
  alt/.code args={<#1>#2#3}{%
    \alt<#1>{\pgfkeysalso{#2}}{\pgfkeysalso{#3}} % \pgfkeysalso doesn't change the path
  },
}

%%%%% draw functions by using tikz
\usepackage{pgfplots}

%colors
\definecolor{tred}{rgb}{0.35, 0.0, 0.0}
\definecolor{tblue}{rgb}{0.0, 0.0, 0.35}
\definecolor{grey}{rgb}{0.4, 0.4, 0.4}
\definecolor{wbgrey}{rgb}{0.9, 0.9, 1}
\definecolor{wrgrey}{rgb}{1, 0.9, 0.9}
\definecolor{wggrey}{rgb}{0.9, 1, 0.9}
\definecolor{periwinkle}{rgb}{0.8, 0.8, 1.0}
\definecolor{purpleheart}{rgb}{0.41, 0.21, 0.61}

%block color
\setbeamercolor{block title}{bg=purpleheart!90, fg=white}

\usepackage{mathtools}
\usepackage[export]{adjustbox}

% set separated appendix numbers
\usepackage{appendixnumberbeamer}

%%% vertical skips
\newcommand{\sk}{\vspace{0.1 in}}
\newcommand{\mk}{\vspace{0.2 in}}
\newcommand{\bk}{\vspace{0.3 in}}
\newcommand{\nk}{\vspace{- 0.1 in}}


\DeclarePairedDelimiter{\ceil}{\lceil}{\rceil}

\newcommand{\scrG}{\mathscr{G}}
\newcommand{\scrF}{\mathscr{F}}

\newtheorem{proposition}[theorem]{Proposition}

% turn off navigation symbols
\setbeamertemplate{navigation symbols}{}

% for appendix
\usepackage{appendixnumberbeamer}

%\usepackage{bm}         % For typesetting bold math (not \mathbold)
%\logo{\includegraphics[height=0.6cm]{caltechslide.pdf}}

%%%%%%%%%%%%%%%%%%%%%%%%%%%%%%%% Here are some custumized commands %%%%%%%%%%%%%%%%%%%%%%%%%%%%%%%%%%%%%%

\newtheorem{axiom}{Axiom}

\newcommand{\plim}{\overset{p}{\longrightarrow}}% converges in probability to.
\newcommand{\rthlim}{\overset{L^r}{\longrightarrow}}% converges in r-th mean
\newcommand{\ip}[1]{\langle #1 \rangle}
\newcommand{\Lim}[1]{\raisebox{0.5ex}{\scalebox{0.8}{$\displaystyle \lim_{#1}\;$}}}

\newcommand{\argmax}{\operatornamewithlimits{argmax}}
\newcommand{\argmin}{\operatornamewithlimits{argmin}}
\newcommand{\bigtimes}{\operatornamewithlimits{\BIGOP{\times}}}

\newcommand{\pr}{\mathbbm{P}}
\newcommand{\ex}{\mathbbm{E}}
\newcommand{\norm}[1]{\left|\left|#1\right|\right|}

% Number systems
\newcommand{\NN}{\mathbb{N}}
\newcommand{\ZZ}{\mathbb{Z}}
\newcommand{\QQ}{\mathbb{Q}}
\newcommand{\RR}{\mathbb{R}}

\newcommand{\GG}{\bm{G}}
\newcommand{\RN}{\bm{N}}
\newcommand{\RM}{\bm{M}}
\newcommand{\RZ}{\bm{Z}}
\newcommand{\bI}{\bm{I}}
\newcommand{\XX}{\bm{X}}

\newcommand{\ff}{\bm{f}}
\newcommand{\xx}{\bm{x}}
\newcommand{\ba}{\bm{a}}
\newcommand{\rz}{\bm{z}}


\newcommand{\idc}{\mathbbm{1}}

\newcommand{\calE}{\mathcal{E}}
\newcommand{\calP}{\mathcal{P}}
\newcommand{\calR}{\mathcal{R}}
\newcommand{\calF}{\mathcal{F}}
\newcommand{\calV}{\mathcal{V}}
\newcommand{\calI}{\mathcal{I}}

\newcommand{\rmu}{\bm{\mu}}
\newcommand{\Rm}{\bm{m}}
\newcommand{\pp}{\bm{p}}
\newcommand{\dd}{\bm{d}}

\newcommand{\DD}{\mathcal{D}}

\definecolor{tred}{RGB}{200,50,80}
\definecolor{tgreen}{RGB}{80,180,80}
\definecolor{tblue}{RGB}{50,50,250}
\definecolor{tpupple}{RGB}{150,50,200}

% reduce space around align
\usepackage{etoolbox}
\newcommand{\zerodisplayskips}{%
  \setlength{\abovedisplayskip}{0.05 in}
  \setlength{\belowdisplayskip}{0.05 in}
  \setlength{\abovedisplayshortskip}{0.05 in}
  \setlength{\belowdisplayshortskip}{0.05 in}}
\appto{\normalsize}{\zerodisplayskips}
\appto{\small}{\zerodisplayskips}
\appto{\footnotesize}{\zerodisplayskips}

\setbeamerfont{caption}{size=\scriptsize}
\setbeamerfont{footnote}{size=\scriptsize}


% include pdf version 1.6
%\pdfoptionpdfminorversion=7 


\usepackage{kotex}
\usepackage{textcomp} % this is to use Korean currency, Won.

\usepackage{scrextend}

\usepackage{fancybox}     
\usepackage{booktabs}              
\usepackage{multirow}

%\newcolumntype{?}{!{\vrule width 1pt}}

\usetikzlibrary{decorations.pathreplacing,angles,quotes} % to use curled braces
\usepackage{changepage} % to move page

%%%% testing environments

%\usepackage{thmtools}
%\usepackage{boiboites} % this requires boiboites.sty in the folder
\usepackage{eshinbox}

\newcounter{thm}
\numberwithin{thm}{section}

\newboxedtheorem[boxcolor=black, background=gray!5, titlebackground=gray!20,
titleboxcolor = black]{thm}{Theorem}{thm}

\newboxedtheorem[boxcolor=black, background=gray!5, titlebackground=gray!20,
titleboxcolor = black]{prp}{Proposition}{thm}

%%%%%%%%%%%%%%%%%%%%%%%%%%%%%%%% START %%%%%%%%%%%%%%%%%%%%%%%%%%%%%%%%%%%%%%

\title[Short title of the talk]{Rationality and Preference Aggregation of Group Decision under Risk}
\author{Syngjoo Choi (SNU), Booyuel Kim (KDIS), Minseon Park (Wisconsin), Yoonsoo Park (KDI), Euncheol Shin (KHU)}
\institute[]
{Presenter: Euncheol Shin (Kyung Hee University) \\
\medskip
{\tt{eshin.econ@khu.ac.kr}}
}
\date{%June, 2017
\today %will show current date.
%324 Alternatively, you can specify a date.
}

%
\begin{document}
%
\begin{frame}
\titlepage
\end{frame}

%%%%%%%%%%%%%%%%%%%%%%%% new frame %%%%%%%%%%%%%%%%%%%%%%%%

\begin{frame}{Introduction}

\begin{itemize}

\item In various contexts, many important decisions are made by groups.
\mk

\item Individual heterogeneity exists in various dimensions:
\sk
	\begin{itemize}
	
	\item Risk preference: risk assessment in environmental policy
	\sk
	
	\item Time preference: household savings and consumption decisions
	\sk

	\item Rationality
	
	\end{itemize}
	\mk
	
\item It is important to understand how individual heterogeneity in a collective influences final outcomes.

\end{itemize}

\end{frame}

%%%%%%%%%%%%%%%%%%%%%%%% new frame %%%%%%%%%%%%%%%%%%%%%%%%

\begin{frame}{Introduction: Research Questions}

\begin{itemize}

\item[1.] Rationality extension:
	\sk
	\begin{itemize}
	
	\item Do rational members make more collectively rational decisions?
	
	\end{itemize}
	\mk

\item[2] Risk preference aggregation:
	\sk
	\begin{itemize}
	
	\item Are individual's risk preferences reflected into that of a group?
	
	\end{itemize}
	\mk
	
\item[3.] Efficiency and welfare:
	\sk
	\begin{itemize}
	
	\item How is the efficiency of group decisions related to individual's rationality and preferences?
	\sk
	
	\item How is social welfare related to individual' rationality and preferences?
	
	\end{itemize}
	
\end{itemize}

\end{frame}

%%%%%%%%%%%%%%%%%%%%%%%% new frame %%%%%%%%%%%%%%%%%%%%%%%%

\begin{frame}{Introduction: Examples of Individual Heterogeneity}

\begin{itemize}

\item Rationality
	\sk
	\begin{itemize}
	
	\item High-income, high-education, men, and young subjects tend more toward utility maximization (Choi et al., 2014).
	
	\end{itemize}
	\mk

\item Risk preference
	\sk
	\begin{itemize}
	
	\item White males are more likely to perceive risks as being smaller (Bickerstaff, 2004; Flynn et al., 1994).
	\sk
	
	\item There is no substantial difference between men and women (Kagel and Roth, 2016).
	\sk
	
	\item Subjects' risk preferences are closer to neutrality when they make decisions on behalf of other participants (Batteux et al., 2017).
	\sk
	
	\item High-power groups adopt a more positive attitude toward potential risks (Anderson and Galinsky, 2006; Magee et al., 2007; Geng et al, 2018).
	
	\end{itemize}
	
\end{itemize}

\end{frame}

%%%%%%%%%%%%%%%%%%%%%%%% new frame %%%%%%%%%%%%%%%%%%%%%%%%

\begin{frame}[noframenumbering]{}

\center
\textbf{Experimental Design and Subjects}

\end{frame}

%%%%%%%%%%%%%%%%%%%%%%%% new frame %%%%%%%%%%%%%%%%%%%%%%%%

\begin{frame}{Experimental Design (Choi et al.,2007; Choi et al., 2014)}
\nk

\begin{figure}
\center
\large 
\begin{tikzpicture}

% axes
\draw[->, thick] (0,0) -- (7,0) node[anchor = west] {$x_r$};
\draw[->, thick] (0,0) -- (0,7) node[anchor = east] {$x_b$};
		
% budget lines
\draw[thick] (6,0) -- (0,3);

% 45 degree line
\draw[dotted, thick] (0,0) -- (4,4) node[anchor = south west] {$x_r = x_b$};
\draw (0.8, 0) node[anchor = south] {$45^\circ$};

% prices
\draw[->, thick] (6,0) -- (6+1/2, 1) node[anchor = west] {$p = (p_r, p_b)$};

% bundles
\draw (2, 2) node {\textbullet};
\draw (2, 2.1) node[anchor = south] {$x$};

\draw (3, 6-3/2*3) node {\textbullet};
\draw (3, 6-3/2*3) node[anchor = north east] {$x'$};

\draw (6, 0) node {\textbullet};
\draw (6, - 0.6) node[anchor = south] {$x''$};

% texts
\visible<1>{
\small{
\draw (3, 7) node[anchor = west] {Two \textbf{equally likely} states: $\textcolor{tred}{R}$ and $\textcolor{tblue}{B}$.};

\draw (3, 6.5) node[anchor = west] {There are two associated Arrow securities.};

\draw (3, 6) node[anchor = west] {$x_r$ is the demand for the security that pays off in state $\textcolor{tred}{R}$.};

\draw (3, 5.5) node[anchor = west] {$x_b$ is the demand for the security that pays off in state $\textcolor{tblue}{B}$.};

\draw (3, 5) node[anchor = west] {Budget constraint: $p_r x_r + p_b x_b = 1$.};
}
}

\visible<2>{
\small{
\draw (3, 7) node[anchor = west] {In this example, $p_b > p_r$.};

\draw (3, 6.5) node[anchor = west] {Risk neutral agent will choose $x''$.};

\draw (3, 6) node[anchor = west] {Extremely risk averse agent will choose $x$.};

\draw (3, 5.5) node[anchor = west] {Intermediately risk averse agent will choose $x'$.};

%\draw (3, 5) node[anchor = west] {$r = \frac{x_r}{x_r + x_b}$};
}
}

\end{tikzpicture}
\end{figure}

\end{frame}

%%%%%%%%%%%%%%%%%%%%%%%% new frame %%%%%%%%%%%%%%%%%%%%%%%%

\begin{frame}{Procedure and Subjects}

\begin{itemize}

\item We conducted the experiment in 12 middle schools in Daegu.
\mk

\item The number of students: 1572.
\mk

\item The number of groups: 786.
\mk

\item The instructions were read by an experimenter in each classroom.
\mk
	
\item Each subject participated in two sessions: individual and group decisions.

\end{itemize}

\end{frame}

%%%%%%%%%%%%%%%%%%%%%%%% new frame %%%%%%%%%%%%%%%%%%%%%%%%

\begin{frame}{Field}

\begin{figure}
\center
\includegraphics[width = 4.0 in]{figures/pblrealshot}
\end{figure}

\end{frame}

%%%%%%%%%%%%%%%%%%%%%%%% new frame %%%%%%%%%%%%%%%%%%%%%%%%

\begin{frame}{Screenshot}

\begin{figure}
\center
\includegraphics[width = 3.8 in]{figures/expshot}
\end{figure}

\end{frame}

%%%%%%%%%%%%%%%%%%%%%%%% new frame %%%%%%%%%%%%%%%%%%%%%%%%

\begin{frame}{Procedure and Subjects}

\begin{itemize}

\item Each round started by having the computer select a budget line randomly from the set of lines that intersect at least one axis at or above $300$ KRW or below $1500$ KRW.
\mk

\item Each session consisted of 18 independent decision rounds.
\mk

\item At the end of each round, the computer randomly selected one of the two states ($\textcolor{tred}{R}$ and $\textcolor{tblue}{B}$).
\mk

\item Subjects were not informed of the state that was selected at the end of each round.

\end{itemize}

\end{frame}

%%%%%%%%%%%%%%%%%%%%%%%% new frame %%%%%%%%%%%%%%%%%%%%%%%%

\begin{frame}{Procedure and Subjects}

\begin{itemize}

\item Two students in the same classroom were randomly matched.
\mk

\item One of the two students was randomly chosen to move to the other partner's desk. 
\mk

\item They made a series of collective decisions by sharing one computer.
\mk

\item We allowed students to discuss how to make decisions for 1 min before starting the second session.
\mk

\item Each subject was paid for he/she earned in a randomly selected round.

\end{itemize}

\end{frame}

%%%%%%%%%%%%%%%%%%%%%%%% new frame %%%%%%%%%%%%%%%%%%%%%%%%

\begin{frame}[noframenumbering]{Example of Choice Data}

\begin{figure}
\center
%\includegraphics[width = 4.5 in]<1>{figures/Figure_Choice_Example1}
\includegraphics[width = 4.5 in]<1>{figures/Figure_Choice_Example2}
\end{figure}

\end{frame}

%%%%%%%%%%%%%%%%%%%%%%%% new frame %%%%%%%%%%%%%%%%%%%%%%%%

\begin{frame}[noframenumbering]{}

\center
\textbf{Result 1: Rationality Extension}

\end{frame}

%%%%%%%%%%%%%%%%%%%%%%%% new frame %%%%%%%%%%%%%%%%%%%%%%%%

\begin{frame}{Measurement: Afriat's Efficiency Index (a.k.a. CCEI)}

\begin{adjustwidth}{-0.5cm}{}
\begin{figure}
\center
\large 
\begin{tikzpicture}

% axes
\draw[->, thick] (0,0) -- (7,0) node[anchor = west] {};
\draw[->, thick] (0,0) -- (0,7) node[anchor = east] {};
		
% budget lines
\draw[thick] (6,0) -- (0,4);
\draw[thick] (0,6) -- (4,0);

% bundles
\draw (3.5, 6-3/2*3.5) node {\textbullet};
\draw (3.5, 6-3/2*3.5) node[anchor = north east] {$x'$};

\draw (6-3/2*3, 3) node {\textbullet};
\draw (6-3/2*3, 3) node[anchor = north east] {$x$};

% rationalization
\visible<2,4>{
\draw[tred, ultra thick, dashed] (0,6-3/2*0.5) -- (4-0.5,0);
%\draw[tblue, ultra thick, dashed] (6-3/2*0.5, 0) -- (0, 4-0.5);
}

\visible<2>{
\draw[decoration={brace, raise=0pt, amplitude=12pt}, xshift=0pt,yshift=0pt, decorate, thick]
  (0,0) -- node[left=1pt,midway,xshift=-0.3cm] {$e_v$} (0,6-3/4);
}

\visible<3->{
\draw[tblue, ultra thick, dashed] (6-3/2*0.88, 0) -- (0, 4-0.88);
}

% texts
\small{
\draw (3, 7) node[anchor = west] {If there are two goods, a choice dataset satisfies the GARP};

\draw (3, 6.5) node[anchor = west] {if and only if it satisfies the WARP.};

\draw (3, 5.5) node[anchor = west] {$x'$ and $x$ violate the GARP.};

\draw (3, 4.5) node[anchor = west] {If the budget line for $x'$ is deflated, the GARP is satisfied.};

\draw (3, 4) node[anchor = west] {Choose $e_v$ for each violation $v$.};

\draw (3, 3) node[anchor = west] {\textbf{Critical cost efficiency index} (CCEI) is defined as the};

\draw (3, 2.5) node[anchor = west] {\textbf{supremum} over all the numbers $e_v$'s.};
}

\end{tikzpicture}
\end{figure}
\end{adjustwidth}

\end{frame}

%%%%%%%%%%%%%%%%%%%%%%%% new frame %%%%%%%%%%%%%%%%%%%%%%%%

\begin{frame}
\frametitle{Measurement: Afriat's Efficiency Index (a.k.a. CCEI)}

\begin{itemize}

\item By definition, CCEI $\in [0,1]$.
\mk

\item The \textcolor{tred}{bigger} CCEI is, the \textcolor{tred}{less} severe violation of GARP.
\mk

\item Basic statistics of individual CCEI:
	\sk
	\begin{itemize}
	
	\item Average: 0.897 (0.136)
	\sk
	
	\item Quantiles: 0.838 (25\%), 0.953 (50\%), 1 (75\%).
	
	\end{itemize}
	\mk 
	
\item Basic statistics of collective CCEI:
	\sk
	\begin{itemize}
	
	\item Average: 0.910 (0.141)
	\sk
	
	\item Quantiles: 0.868 (25\%), 0.981 (50\%), 1 (75\%).
	
	\end{itemize}

\end{itemize}

\end{frame}

%%%%%%%%%%%%%%%%%%%%%%%% new frame %%%%%%%%%%%%%%%%%%%%%%%%

\begin{frame}{Rationality Extension: Research Question}

\center
Individual Rationality $\uparrow$ $\Rightarrow$ Collective Rationality $\uparrow$?

\end{frame}

%%%%%%%%%%%%%%%%%%%%%%%% new frame %%%%%%%%%%%%%%%%%%%%%%%%

\begin{frame}{Rationality Extension: First-Order Stochastic Dominance}

\begin{figure}
\center
\includegraphics[width = 4.5 in]{figures/Figure1_CCEI_90}
%\caption{Computerized experiment in a classroom}
\end{figure}

\end{frame}

%%%%%%%%%%%%%%%%%%%%%%%% new frame %%%%%%%%%%%%%%%%%%%%%%%%

\begin{frame}
\frametitle{Rationality Extension: First-Order Stochastic Dominance}

\begin{itemize}

\item We do a series of Kolmogorov-Smirnov tests:
	\sk
	\begin{align*}
	H_0: F_{\text{group $i$}} (X) = F_{\text{group $j$}} (X) \quad \text{for all values of $X$.}
	\end{align*}
\sk \nk

\item Test statistic: $D_{ij} = \sup_{x \in X} \vert \vert F_{\text{group $i$}} (x) - F_{\text{group $j$}} (x) \vert \vert$.
\mk

\item (Low, Low) v.s. (High, High): 0.17
\sk
	\begin{itemize}
	
	\item The corresponding p-value is 0.01.

	\end{itemize}
	\mk
	
\item (Low, High) v.s. (High, High): 0.21
\sk
	\begin{itemize}
	
	\item The corresponding p-value is 0.00.

	\end{itemize}

\end{itemize}

\end{frame}

%%%%%%%%%%%%%%%%%%%%%%%% new frame %%%%%%%%%%%%%%%%%%%%%%%%

\begin{frame}{Rationality Extension: Econometric Analysis}

\vspace{0.10 in}
%\hspace{-0.15 in}
\centering \changefontsizes{8pt}
\begin{tabular}{l  lc lc lc}
\toprule
\multirow{2}{*}{\textbf{Collective CCEI}}	& & \multicolumn{5}{c}{\textbf{Coefficient}}  \\
		        \cline{3-7}
																& & Model 1 & & Model 2 & & Model 3 \\
\hline
\hline
\multirow{2}{*}{CCEI\underline{{ }{ }}Max}  & &  $0.368^{***}$  & & $0.327^{***}$  & & $0.302^{***}$   \\
                                                                        & & $(0.083)$         & & $(0.074)$          & & $(0.089)$          \\
\multirow{2}{*}{CCEI\underline{{ }{ }}Distance}  & & $-0.277^{***}$ & & $-0.250^{***}$  & &  $-0.233^{***}$  \\
                                                                                & & $(0.056)$        & & $(0.053)$           & & $(0.058)$       \\
\hline
\multirow{2}{*}{Risk\underline{{ }{ }}Aversion\underline{{ }{ }}Max}  & &      & & $-0.189^{***}$  & &  $-0.172^{**}$    \\
                                                                                                                 & &      & & $(0.056)$      & & $(0.070)$    \\
\multirow{2}{*}{Risk\underline{{ }{ }}Aversion\underline{{ }{ }}Distance}  & &       & & $0.087^{*}$     & &  $0.093^{*}$   \\
                             																			         & &      & & $(0.048)$   & & $(0.055)$   \\
\hline
\multirow{2}{*}{Math\underline{{ }{ }}Score\underline{{ }{ }}Max}  & &     & &     & &  $0.012^{**}$  \\
                                                                                                             & &     & &     & &  $(0.005)$     \\
\multirow{2}{*}{Math\underline{{ }{ }}Distance}  & &        & &         & &  $-0.010^{**}$      \\
                                                						        & &          & &        & &  $(0.005)$     \\
\multirow{2}{*}{Constant}      & & $0.582^{***}$   & & $0.679^{***}$    & & $0.664^{***}$  \\
                                                 	  & &  $(0.077)$        & &  $(0.070)$         & &  $(0.084)$  \\
\hline
\hline
Class Fixed Effect           & & Yes         & & Yes         & & Yes            \\
Individual Characteristics      & & No         & & No         & & Yes         \\
School Characteristics           & & No         & &  No         & & Yes          \\
Friendship                              & & No         & &  No         & & Yes        \\
Observations                  & & $786$ & & $786$ & & $786$    \\
$R$-squared                  & & $0.200$  & & $0.212$  & & $0.235$  \\
\hline
\multicolumn{7}{r}{\small{$*$Throughout the paper, we clustered standard error in class level.}} \\
\bottomrule 
\end{tabular}

\end{frame}  

%%%%%%%%%%%%%%%%%%%%%%%% new frame %%%%%%%%%%%%%%%%%%%%%%%%

\begin{frame}[noframenumbering]{}

\center
\textbf{Result 2: Preference Aggregation}

\end{frame}

%%%%%%%%%%%%%%%%%%%%%%%% new frame %%%%%%%%%%%%%%%%%%%%%%%%

\begin{frame}{Measurement: Indifference Curves}
\nk

\begin{figure}
\center
\large 
\begin{tikzpicture}

% axes
\draw[->, thick] (0,0) -- (7,0) node[anchor = west] {$x_r$};
\draw[->, thick] (0,0) -- (0,7) node[anchor = east] {$x_b$};
		
% budget lines
\draw[thick] (6,0) -- (0,3);

% 45 degree line
\draw[dotted, thick] (0,0) -- (4,4) node[anchor = south west] {};
\draw (0.8, 0) node[anchor = south] {$45^\circ$};

\visible<1->{
% choice
\draw (3.7, 1.15) node {\textbullet};
\draw (3.5, 1.15) node[anchor = south west] {$x = (x_{\text{cheap}} , x_{\text{expensive}} )$};
\draw[->, thick] (6,0) -- (6+1/2, 1) node[anchor = west] {$p = (p_r, p_b)$};
}

% texts
\visible<1->{
\small{
\draw (2, 6.5) node[anchor = west] {We non-parametrically measure the risk preference by a ratio:};

\draw (3, 5.5) node[anchor = west] {risk aversion (RA) = $\frac{x_{\text{expensive}} }{ x_{\text{expensive}} + x_{\text{cheap}} }$};
}
}

\end{tikzpicture}
\end{figure}

\end{frame}

%%%%%%%%%%%%%%%%%%%%%%%% new frame %%%%%%%%%%%%%%%%%%%%%%%%

\begin{frame}
\frametitle{Measurement: Risk Preferences}

\begin{itemize}

\item By definition, $\text{RA} \in [0,1]$.
\mk

\item The \textcolor{tred}{bigger} ratio is, the \textcolor{tred}{higher} risk aversion.
\mk

\item Basic statistics of individual RA:
	\sk
	\begin{itemize}
	
	\item Average: 0.324 (0.132).
	\sk
	
	\item Quantiles: 0.213 (25\%)	, 0.310 (50\%), 0.390 (75\%), 0.499 (99\%).
	
	\end{itemize}
	\mk
	
\item Basic statistics of collective RA:
	\sk
	\begin{itemize}
	
	\item Average: 0.298	 (0.139).
	\sk
	
	\item Quantiles: 0.255 (25\%), 0.348 (50\%), 0.413 (75\%), 0.497 (99\%).
	
	\end{itemize}

\end{itemize}

\end{frame}

%%%%%%%%%%%%%%%%%%%%%%%% new frame %%%%%%%%%%%%%%%%%%%%%%%%

\begin{frame}{Risk Preference Aggregation: Research Question}

\center
Individual risk aversion $\uparrow$ $\Rightarrow$ Collective risk aversion $\uparrow$?

\end{frame}

%%%%%%%%%%%%%%%%%%%%%%%% new frame %%%%%%%%%%%%%%%%%%%%%%%%

\begin{frame}{Preference Aggregation: FOSD by Relative Ratio}

\begin{figure}
\center
\includegraphics[width = 4.5 in]{figures/Figure2A_riskaversion}
%\caption{Computerized experiment in a classroom}
\end{figure}

\end{frame}

%%%%%%%%%%%%%%%%%%%%%%%% new frame %%%%%%%%%%%%%%%%%%%%%%%%

\begin{frame}{Preference Aggregation: Econometric Analysis}

\vspace{0.10 in}
%\hspace{-0.15 in}
\centering \changefontsizes{8pt}
\begin{tabular}{l  lc lc lc}
\toprule
\multirow{2}{*}{\textbf{Collective Risk Aversion}}	& & \multicolumn{5}{c}{\textbf{Coefficient}}  \\
		        \cline{3-7}
																& & Model 1 & & Model 2 & & Model 3 \\
\hline
\hline
\multirow{2}{*}{Risk\underline{{ }{ }}Aversion\underline{{ }{ }}Max}  & &  $0.792^{***}$  & & $0.808^{***}$  & & $0.759^{***}$ \\
                                                                                                                 & &  $(0.066)	$  & & $(0.063)$      & & $(0.073)$    \\
\multirow{2}{*}{Risk\underline{{ }{ }}Aversion\underline{{ }{ }}Distance}  & &  $-0.421^{***}$  & & $-0.432^{***}$   & &  $-0.434^{***}$   \\
                             																			         & &  $(0.053)$  & & $(0.053)$   & & $(0.062)$   \\
\hline
\multirow{2}{*}{CCEI\underline{{ }{ }}Max}  & &      & & $0.165^{**}$  & & $0.196^{***}$   \\
                                                                        & &       & & $(0.064)$          & & $(0.071)$          \\
\multirow{2}{*}{CCEI\underline{{ }{ }}Distance}  & &    & & $0.014$  & &  $0.040$  \\
                                                                                & &    & & $(0.033)$           & & $(0.045)$       \\
\hline
\multirow{2}{*}{Math\underline{{ }{ }}Score\underline{{ }{ }}Max}  & &     & &     & &  $0.000$  \\
                                                                                                             & &     & &     & &  $(0.005)$     \\
\multirow{2}{*}{Math\underline{{ }{ }}Distance}  & &        & &         & &  $0.005$      \\
                                                						        & &          & &        & &  $(0.004)$     \\
\multirow{2}{*}{Constant}      & & $0.004$   & & $-0.156^{**}$    & & $-0.175^{**}$  \\
                                                 	  & &  $(0.027)$        & &  $(0.063)$         & &  $(0.077)$  \\
\hline
\hline
Class Fixed Effect                  & & Yes         & & Yes         & & Yes            \\
Individual Characteristics      & & No         & & No         & & Yes         \\
School Characteristics           & & No         & &  No         & & Yes          \\
Friendship                              & & No         & &  No         & & Yes        \\
Observations                  & & $786$ & & $786$ & & $786$    \\
$R$-squared                  & & $0.372$  & & $0.378$  & & $0.382$  \\
\bottomrule 
\end{tabular}

\end{frame}

%%%%%%%%%%%%%%%%%%%%%%%% new frame %%%%%%%%%%%%%%%%%%%%%%%%

\begin{frame}[noframenumbering]{}

\center
\textbf{Result 3: Efficiency and Welfare}

\end{frame}

%%%%%%%%%%%%%%%%%%%%%%%% new frame %%%%%%%%%%%%%%%%%%%%%%%%

\begin{frame}{Measurement: Idea}

\begin{itemize}

\item We analyze the \textbf{quality} of collective decisions as a function of the degrees of rationality and preference alignment.
\mk

\item Idea: 
\sk
	\begin{itemize}
	
	\item We consider a class of utility functions over lotteries.
	\sk
	
	\item For each subject, we estimate the utility function parametrically.
	\sk
	
	\item We characterize a set of Pareto efficient choices.
	\sk
	
	\item For collective choices which are \textbf{not} Pareto efficient, we measure the degree of welfare loss.
	
	\end{itemize}

\end{itemize}

\end{frame}

%%%%%%%%%%%%%%%%%%%%%%%% new frame %%%%%%%%%%%%%%%%%%%%%%%%

\begin{frame}{Measurement: Utility Estimation}
\nk

\begin{figure}
\center
\large 
\begin{tikzpicture}

% axes
\draw[->, thick] (0,0) -- (7,0) node[anchor = west] {$x_r$};
\draw[->, thick] (0,0) -- (0,7) node[anchor = east] {$x_b$};
		
% budget lines
\draw[thick] (6,0) -- (0,3);

% 45 degree line
\draw[dotted, thick] (0,0) -- (4,4) node[anchor = south west] {};
\draw (0.8, 0) node[anchor = south] {$45^\circ$};

\visible<2>{
% indifference curve
\draw[thick, tred] (2.3, 2.3) to [out=-45, in=170] (6,0.4) node[anchor = south west] {};
\draw[thick, tred] (2.3, 2.3) to [out=135, in=-80] (0.4,6) node[anchor = south west] {};
% choice
\draw (4, 1) node {\textbullet};
\draw (4, 1) node[anchor = south west] {$x_{\text{EU}}$};
}

\visible<3>{
% indifference curve
\draw[thick, tblue] (2.5, 2.5) to [out=-65, in=170] (6,0.5) node[anchor = south west] {};
\draw[thick, tblue] (2.5, 2.5) to [out=155, in=-80] (0.5,6) node[anchor = south west] {};
% choice
\draw (3.7, 1.15) node {\textbullet};
\draw (3.8, 1.15) node[anchor = north east] {$x_{\text{DAU}}$};
}

% texts
\visible<1->{
\small{
\draw (2, 7.5) node[anchor = west] {We assume a CARA utility function: $u(x) = - e^{-\rho x} / \rho$.};

\draw (2, 7) node[anchor = west] {We consider two utility functions over lotteries.};
}
}

% texts
\visible<2>{
\small{
\draw (2, 6) node[anchor = west] {Expected utility (EU) by von Neumann and Morgenstern (1953):};

\draw (3, 5.5) node[anchor = west] {$U(x) = \frac{1}{2} u(x_{\text{max}}) + \frac{1}{2} u(x_{\text{min}}).$};

\draw (2, 5) node[anchor = west] {The red line is an indifference curve for the EU.};
}
}

% texts
\visible<3>{
\small{
\draw (2, 6) node[anchor = west] {Disappointment aversion utility (DAU) by Gul (1991):};

\draw (3, 5.5) node[anchor = west] {$U(x) = \alpha u(x_{\text{max}}) + (1-\alpha) u(x_{\text{min}})$.};

\draw (2, 5) node[anchor = west] {$\rho$ determines the curvature of the curve.};

\draw (2, 4.5) node[anchor = west] {$\alpha$ determines the shape of the kink.};
}
}

\end{tikzpicture}
\end{figure}

\end{frame}

%%%%%%%%%%%%%%%%%%%%%%%% new frame %%%%%%%%%%%%%%%%%%%%%%%%

\begin{frame}{Measurement: Utility Estimation}

\begin{itemize}

\item We restrict our attention to a CARA utility function over outcomes.
\mk

\item We consider two different types of utility function over lotteries:
	\sk
	\begin{itemize}
	\item Expected utility (EU)
	\sk
	
	\item Disappointment aversion utility (DAU).
	\end{itemize}
	\mk

\item We estimate $\rho$ and $\beta$ simultaneously by using a combination of a bootstrapping and the non-linear least square (NLLS) methods:
	\sk
	\begin{itemize}
	
	\item[1] Find subsample of size 18 with replacement.
	\sk
	
	\item[2] For given subsample, estimate $\alpha$ and $\rho$  by NLLS.
	\sk
	
	\item[3] Repeat the above for 250 times.
	\sk
	
	\item[4] If $0.5 \in [\alpha_{2.5}, \alpha_{97.5}]$, then set $\alpha = 0.5$ as an EU.
	\sk
	
	\item[4'] Otherwise, set $\alpha = \overline{\alpha}$ as a DAU.
	
	\end{itemize}

\end{itemize}

\end{frame}

%%%%%%%%%%%%%%%%%%%%%%%% new frame %%%%%%%%%%%%%%%%%%%%%%%%

\begin{frame}{Measurement: Efficiency and Welfare Loss}
\nk

\begin{figure}
\center
\large 
\begin{tikzpicture}

% axes
\draw[->, thick] (0,0) -- (7,0) node[anchor = west] {$x_r$};
\draw[->, thick] (0,0) -- (0,7) node[anchor = east] {$x_b$};
		
% budget lines
\draw[thick] (6,0) -- (0,3);

\visible<1->{
% indifference curve1
\draw[thick, tred] (2.2, 1.9) to [out=-20, in=180] (6,1) node[anchor = south west] {};
\draw[thick, tred] (2.2, 1.9) to [out=150, in=-90] (0.2,6) node[anchor = south west] {};
% choice
\draw (2.2, 1.9) node {\textbullet};
\draw (2.2, 1.9) node[anchor = south west] {$x^*$};
}

\visible<1->{
% indifference curve2
\draw[thick, tblue] (5, 0.48) to [out=-20, in=170] (6,0.2) node[anchor = south west] {};
\draw[thick, tblue] (5, 0.48) to [out=140, in=-80] (1.5,5) node[anchor = south west] {};
% choice
\draw (5, 0.48) node {\textbullet};
\draw (5, 0.48) node[anchor = south west] {$y^*$};
}

\visible<2->{
\small
% budget lines
\draw[ultra thick] (2.2,1.9) -- (5,0.48);
\draw (5, 6) node[anchor = south] {The set of Pareto efficient choices};
\draw[thick, ->] (5,6) -- (3.5,1.3);
}

\end{tikzpicture}
\end{figure}

\end{frame}

%%%%%%%%%%%%%%%%%%%%%%%% new frame %%%%%%%%%%%%%%%%%%%%%%%%

\begin{frame}{Measurement: Efficiency and Welfare Loss}
\nk

\begin{figure}
\center
\large 
\begin{tikzpicture}

% axes
\draw[->, thick] (0,0) -- (7,0) node[anchor = west] {$U_1$};
\draw[->, thick] (0,0) -- (0,7) node[anchor = east] {$U_2$};

\visible<1->{
% feasible utilities
\draw[thick] (0, 0) to [out=2, in=-90] (6,2) to [out=90, in=-45] (4.8,4.8);
\draw[thick] (0, 0) to [out=88, in=180] (2,6) to [out=0, in=135] (4.8,4.8);
\draw[ultra thick] (6,2) to [out=90, in=-45] (4.8,4.8);
\draw[ultra thick] (2,6) to [out=0, in=135] (4.8,4.8);

% guild lines
\draw[dotted, thick] (6,0) -- (6,7) node[anchor = south west] {};
\draw (6, 0) node[anchor = north] {$U_1^{\max}$};
%\draw[dotted, thick] (2,0) -- (2,7) node[anchor = south west] {};
%\draw (0, 0) node[anchor = north] {$U_1^{min}$};

\draw[dotted, thick] (0,6) -- (7,6) node[anchor = south west] {};
\draw (0, 6) node[anchor = east] {$U_2^{\max}$};
%\draw[dotted, thick] (0,2) -- (7,2) node[anchor = south west] {};
\draw (0, 0) node[anchor = north east] {$(0,0)$};
}

\visible<2->{
% choice
\draw (5, 0.38) node {\textbullet};
\draw[dotted, thick] (5,0) -- (5,7) node[anchor = south west] {};
\draw[dotted, thick] (0,0.38) -- (7,0.38) node[anchor = south west] {};
\draw (5, 0.38) node[anchor = south east] {$(U_1^o, U_2^o)$};
}


\visible<3->{
\draw[decoration={brace, raise=0pt, amplitude=8pt, mirror}, xshift=0pt,yshift=0pt, decorate, ultra thick, tblue]
(0,0) -- node[below=1pt,midway,yshift=-0.1cm] {$(a)$} (6,0);
\draw[decoration={brace, raise=0pt, amplitude=8pt}, xshift=0pt,yshift=0pt, decorate, ultra thick, tblue]
(5,0.38) -- node[above=1pt,midway,yshift=0.1cm] {$(b)$} (6,0.38);
}

\visible<4->{
\draw[decoration={brace, raise=0pt, amplitude=8pt}, xshift=0pt,yshift=0pt, decorate, ultra thick, tred]
(0,0) -- node[left=1pt,midway,xshift=-0.1cm] {$(c)$} (0,6);
\draw[decoration={brace, raise=0pt, amplitude=8pt, mirror}, xshift=0pt,yshift=0pt, decorate, ultra thick, tred]
(5,0.38) -- node[right=1pt,midway,xshift=0.1cm] {$(d)$} (5,6);
}


\end{tikzpicture}
\end{figure}

\end{frame}

%%%%%%%%%%%%%%%%%%%%%%%% new frame %%%%%%%%%%%%%%%%%%%%%%%%

\begin{frame}{Efficiency and Welfare: Measurement}

\begin{itemize}

\item We focus on the group choices which are not Pareto efficient (60\%).
\mk

\item For those choices, we measure \textbf{welfare loss} of a group as
	\sk
	\begin{align*}
	\text{Welfare Loss} = \frac{1}{18} \sum_{k = 1}^{18} \frac{1}{2} \sum_{i = 1}^2 \frac{U_i(x_{ikb}) - U_i(x_{ick})}{U_i(x_{ikb}) - U_i(x_{ikw})},
	\end{align*}
	\sk
	where
	\sk
	\begin{itemize}
	
	\item $x_{ick}$: group choice in $k$-th round
	\sk
	
	\item $x_{ikb}$: member $i$'s best choice in $k$-th round
	\sk
	
	\item $x_{ikw}$: member $i$'s worst choice in $k$-th round.
	
	\end{itemize}
	\mk
	
\item By definition, $\text{Welfare Loss} \in [0,1]$.

\end{itemize}

\end{frame}

%%%%%%%%%%%%%%%%%%%%%%%% new frame %%%%%%%%%%%%%%%%%%%%%%%%

\begin{frame}{Efficiency and Welfare: Distribution of Welfare Loss}

\begin{figure}
\center
\includegraphics[width = 4.5 in]{figures/Figure3A_ULoss}
%\caption{Computerized experiment in a classroom}
\end{figure}

\end{frame}

%%%%%%%%%%%%%%%%%%%%%%%% new frame %%%%%%%%%%%%%%%%%%%%%%%%

\begin{frame}{Efficiency and Welfare: Research Question}

\center
How is the welfare loss related to individual rationality and risk preference?

\end{frame}

%%%%%%%%%%%%%%%%%%%%%%%% new frame %%%%%%%%%%%%%%%%%%%%%%%%

\begin{frame}{Welfare: FOSD by Group Rationality}

\begin{figure}
\center
\includegraphics[width = 4.5 in]{figures/Figure4A_ULoss_ccei}
%\caption{Computerized experiment in a classroom}
\end{figure}

\end{frame}

%%%%%%%%%%%%%%%%%%%%%%%% new frame %%%%%%%%%%%%%%%%%%%%%%%%

\begin{frame}{Welfare: Econometric Analysis}

\vspace{0.055 in}
%\hspace{-0.15 in}
\centering \changefontsizes{7.5 pt}
\begin{tabular}{l  lc lc lc lc}
\toprule
\multirow{2}{*}{\textbf{Group Inefficiency}}	& & \multicolumn{7}{c}{\textbf{Coefficient}}  \\
		        \cline{3-9}
																& & Model 1 & & Model 2 & & Model 3 & & Model 4 \\
\hline
\hline
\multirow{2}{*}{CCEI\underline{{ }{ }}Group}  & &  $-0.571^{***}$  & & $-0.503^{***}$  & & $-0.527^{***}$ & & $-0.414^{***}$  \\
                                                                        & & $(0.042)$         & & $(0.039)$          & & $(0.043)$  & & $(0.078)$   \\
\hline
\multirow{2}{*}{CCEI\underline{{ }{ }}Max}  & &      & & $-0.296^{***}$  & & $-0.242^{***}$ & & $-0.692$  \\
                                                                        & &       & & $(0.045)$          & & $(0.051)$          & & $(0.671)$  \\
\multirow{2}{*}{CCEI\underline{{ }{ }}Distance}  & &         & & $0.165^{***}$  & &  $0.178^{***}$  & & $0.289$  \\
                                                                                & &        & & $(0.042)$           & & $(0.052)$   & & $(0.257)$  \\
\hline
\multirow{2}{*}{Risk\underline{{ }{ }}Aversion\underline{{ }{ }}Max}  & &    & & $-0.009$  & &  $0.023$   & & $-0.024$  \\
                                                                                                                 & &    & & $(0.056)$      & & $(0.063)$  & & $(0.089)$  \\
\multirow{2}{*}{Risk\underline{{ }{ }}Aversion\underline{{ }{ }}Distance}  & &     & & $-0.057^{*}$  & &  $-0.073^{*}$  & &$-0.123^{*}$ \\
                             																			         & &      & & $(0.031)$   & & $(0.040)$ & & $(0.066)$  \\
\hline
\multirow{2}{*}{Math\underline{{ }{ }}Score\underline{{ }{ }}Max}  & &     & &     & &  $0.002$  & & $0.008$  \\
                                                                                                             & &     & &     & &  $(0.005)$   & & $(0.007)$  \\
\multirow{2}{*}{Math\underline{{ }{ }}Distance}  & &        & &         & &  $-0.004$  & & $-0.010$    \\
                                                						        & &          & &        & &  $(0.003)$  & & $(0.006)$   \\
\multirow{2}{*}{Constant}      & & $0.651^{***}$   & & $0.866^{***}$    & & $0.807^{***}$  & & $1.154^{*}$ \\
                                                 	  & &  $(0.038)$        & &  $(0.048)$         & &  $(0.061)$  & & $(0.653)$  \\
\hline
\hline
Class Fixed Effect                  & & Yes         & & Yes         & & Yes    & & Yes          \\
Individual Characteristics      & & No         & & No         & & Yes    & & Yes       \\
School Characteristics           & & No         & &  No         & & Yes    & & Yes        \\
Friendship                              & & No         & &  No         & & Yes    & & Yes      \\
Observations                  & & $786$ & & $786$ & & $786$  & & $274$      \\
$R$-squared                  & & $0.442$  & & $0.487$  & & $0.497$ & & $0.436$    \\
\bottomrule 
\end{tabular}

\end{frame}

%%%%%%%%%%%%%%%%%%%%%%%% new frame %%%%%%%%%%%%%%%%%%%%%%%%

\begin{frame}{Conclusion}

\begin{itemize}

\item We measure rationality and risk preference in individual and group levels.
\mk

\item We observe rationality extension and preference aggregation.
\mk

\item We develop a measure of efficiency and utility loss of group decisions.
\mk

\item We find that
	\sk
	\begin{itemize}
	\item Rational groups are more likely to make efficient decisions.
	\sk
	
	\item Preference-aligned individuals need not make efficient decisions.
	\end{itemize}
	
\end{itemize}

\end{frame}

%%%%%%%%%%%%%%%%%%%%%%%% new frame %%%%%%%%%%%%%%%%%%%%%%%%

\begin{frame}{Conclusion}

\begin{itemize}

\item Our main findings are robust with respect to
	\sk
	\begin{itemize}
	\item another rationality measure (Varian's efficiency index)
	\sk
	
	\item other cutoff values of CCEI (0.99 or 0.95)
	\sk
	
	\item another measure of risk preferences (risk premium).
	
	\end{itemize}
	
\end{itemize}

\end{frame}

\appendix

%%%%%%%%%%%%%%%%%%%%%%%% new frame %%%%%%%%%%%%%%%%%%%%%%%%

\begin{frame}{}

\center
\textbf{Robustness}

\end{frame}

%%%%%%%%%%%%%%%%%%%%%%%% new frame %%%%%%%%%%%%%%%%%%%%%%%%

\begin{frame}{Varian's Efficiency Index}

\begin{itemize}

\item Varian modifies CCEI by allowing $e_k$ to vary across the different price vectors.
\mk

\item Consider a vector $\theta = (e^k)_{k=1}^K$ of numbers in $[0,1]$, one for each observation.
\mk

\item Define the binary relation $R_\theta$ as $x^k R_\theta x^l$ if $e^k p^k \cdot x^k \geq p^k \cdot x^l$. Let $P_\theta$ be the corresponding strict relation.
\mk

\item There is a set $\Theta$ of vectors $\theta$ such that the corresponding $\langle R_\theta, P_\theta \rangle$ satisfies GARP.
\mk

\item Varian's efficiency index (VEI) is the closest distance of a vector $\theta$ to the unit vector ($e_k = 1$ for all $k$), among those $\theta$ for which the preference pair is acyclic:
	\begin{align*}
	\text{VEI} = \inf \Big\{ \vert \vert 1 - \theta \vert \vert \big\vert \langle R_\theta, P_\theta \rangle \text{ is acyclic} \Big\}.
	\end{align*}

\end{itemize}

\end{frame}

%%%%%%%%%%%%%%%%%%%%%%%% new frame %%%%%%%%%%%%%%%%%%%%%%%%

\begin{frame}{Disappointment Aversion Utility}

\begin{itemize}

\item A functional form:
	\begin{align*}
	u(x_1, x_2) = u(\max \{ x_1, x_2 \}) + (1-\gamma) u(\min \{x_1, x_2 \}),
	\end{align*}
	where $\gamma = 1/(2+\beta)$.
	\mk

\item $\beta > 0$ represents the \textbf{disappointment aversion}: the better outcome is under-weighted relative to the objective probability.
\mk
	
\item $\beta \in (-1, 0)$ represents \textbf{elation seeking}: the better outcome is over-weighted relative to the objective probability.
\mk

\item Of course, this utility function is aligned with the first-order stochastic dominance relationships between lotteries.

\end{itemize}

\end{frame}

%%%%%%%%%%%%%%%%%%%%%%%% new frame %%%%%%%%%%%%%%%%%%%%%%%%

\begin{frame}{Environmental Economics}

\begin{itemize}

\item Textbooks:
\mk
\center
\includegraphics[height= 2.5 in]<1>{figures/envecon}
\includegraphics[height = 2.5 in]<2>{figures/intpubecon}
\hspace{0.2 in}
\includegraphics[height = 2.5 in]<1>{figures/powersysecon}
\includegraphics[height = 2.5 in]<2>{figures/pubecon}
%\hspace{0.2 in}

\end{itemize}

\end{frame}

%%%%%%%%%%%%%%%%%%%%%%%% new frame %%%%%%%%%%%%%%%%%%%%%%%%

%\begin{frame}{Introduction to Experimental Economics}

%\center
%\includegraphics[height = 3 in]<1>{pfigs/dan}
%\includegraphics[height = 3 in]<2>{pfigs/roth}
%\includegraphics[height = 3 in]<3>{pfigs/thaler1}
%\includegraphics[height = 3 in]<4>{pfigs/thaler2}

%\end{frame}

\end{document}


%%%%%% Inventory

%\center
%\includegraphics[width = 1.8 in]{textbook}
