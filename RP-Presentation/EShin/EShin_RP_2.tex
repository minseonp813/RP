\documentclass[10pt, xcolor={dvipsnames,table}]{beamer}
%\documentclass[11pt,handout]{beamer} % for handout or editing

\usetheme{Madrid}
\usecolortheme{dolphin} % Simple and clean template
\setbeamertemplate{footline}[frame number]
\usefonttheme{professionalfonts}

% reduce the margins
\setbeamersize{text margin left = 0.25 in}
\setbeamersize{text margin right = 0.25 in}

%
\usepackage{graphicx}
\usepackage{amssymb}
\usepackage{amsfonts}
\usepackage{amsmath}
\usepackage{amsthm}
\usepackage{mathrsfs}
\usepackage{color}
\usepackage{setspace}
\usepackage{bbm}
\usepackage{subfigure}

%without auto-caption labeling
\usepackage{stackengine}

%%%%% for graphics in latex (e.g. arrows)
\usepackage{tikz}
\usetikzlibrary{arrows,chains,matrix,positioning,scopes,automata}
\tikzset{
  invisible/.style={opacity=0},
  visible on/.style={alt={#1{}{invisible}}},
  alt/.code args={<#1>#2#3}{%
    \alt<#1>{\pgfkeysalso{#2}}{\pgfkeysalso{#3}} % \pgfkeysalso doesn't change the path
  },
}

%%%%% draw functions by using tikz
\usepackage{pgfplots}

%colors
\definecolor{tred}{rgb}{0.35, 0.0, 0.0}
\definecolor{tblue}{rgb}{0.0, 0.0, 0.35}
\definecolor{grey}{rgb}{0.4, 0.4, 0.4}
\definecolor{wbgrey}{rgb}{0.9, 0.9, 1}
\definecolor{wrgrey}{rgb}{1, 0.9, 0.9}
\definecolor{wggrey}{rgb}{0.9, 1, 0.9}
\definecolor{periwinkle}{rgb}{0.8, 0.8, 1.0}
\definecolor{purpleheart}{rgb}{0.41, 0.21, 0.61}

%block color
\setbeamercolor{block title}{bg=purpleheart!90, fg=white}

\usepackage{mathtools}
\usepackage[export]{adjustbox}

% set separated appendix numbers
\usepackage{appendixnumberbeamer}

%%% vertical skips
\newcommand{\sk}{\vspace{0.1 in}}
\newcommand{\mk}{\vspace{0.2 in}}
\newcommand{\bk}{\vspace{0.3 in}}
\newcommand{\nk}{\vspace{- 0.1 in}}


\DeclarePairedDelimiter{\ceil}{\lceil}{\rceil}

\newcommand{\scrG}{\mathscr{G}}
\newcommand{\scrF}{\mathscr{F}}

\newtheorem{proposition}[theorem]{Proposition}

% turn off navigation symbols
\setbeamertemplate{navigation symbols}{}

% for appendix
\usepackage{appendixnumberbeamer}

%\usepackage{bm}         % For typesetting bold math (not \mathbold)
%\logo{\includegraphics[height=0.6cm]{caltechslide.pdf}}

%%%%%%%%%%%%%%%%%%%%%%%%%%%%%%%% Here are some custumized commands %%%%%%%%%%%%%%%%%%%%%%%%%%%%%%%%%%%%%%

\newtheorem{axiom}{Axiom}

\newcommand{\plim}{\overset{p}{\longrightarrow}}% converges in probability to.
\newcommand{\rthlim}{\overset{L^r}{\longrightarrow}}% converges in r-th mean
\newcommand{\ip}[1]{\langle #1 \rangle}
\newcommand{\Lim}[1]{\raisebox{0.5ex}{\scalebox{0.8}{$\displaystyle \lim_{#1}\;$}}}

\newcommand{\argmax}{\operatornamewithlimits{argmax}}
\newcommand{\argmin}{\operatornamewithlimits{argmin}}
\newcommand{\bigtimes}{\operatornamewithlimits{\BIGOP{\times}}}

\newcommand{\pr}{\mathbbm{P}}
\newcommand{\ex}{\mathbbm{E}}
\newcommand{\norm}[1]{\left|\left|#1\right|\right|}

% Number systems
\newcommand{\NN}{\mathbb{N}}
\newcommand{\ZZ}{\mathbb{Z}}
\newcommand{\QQ}{\mathbb{Q}}
\newcommand{\RR}{\mathbb{R}}

\newcommand{\GG}{\bm{G}}
\newcommand{\RN}{\bm{N}}
\newcommand{\RM}{\bm{M}}
\newcommand{\RZ}{\bm{Z}}
\newcommand{\bI}{\bm{I}}
\newcommand{\XX}{\bm{X}}

\newcommand{\ff}{\bm{f}}
\newcommand{\xx}{\bm{x}}
\newcommand{\ba}{\bm{a}}
\newcommand{\rz}{\bm{z}}


\newcommand{\idc}{\mathbbm{1}}

\newcommand{\calE}{\mathcal{E}}
\newcommand{\calP}{\mathcal{P}}
\newcommand{\calR}{\mathcal{R}}
\newcommand{\calF}{\mathcal{F}}
\newcommand{\calV}{\mathcal{V}}
\newcommand{\calI}{\mathcal{I}}

\newcommand{\rmu}{\bm{\mu}}
\newcommand{\Rm}{\bm{m}}
\newcommand{\pp}{\bm{p}}
\newcommand{\dd}{\bm{d}}

\newcommand{\DD}{\mathcal{D}}

\definecolor{tred}{RGB}{200,50,80}
\definecolor{tgreen}{RGB}{80,180,80}
\definecolor{tblue}{RGB}{50,50,250}
\definecolor{tpupple}{RGB}{150,50,200}

% reduce space around align
\usepackage{etoolbox}
\newcommand{\zerodisplayskips}{%
  \setlength{\abovedisplayskip}{0.05 in}
  \setlength{\belowdisplayskip}{0.05 in}
  \setlength{\abovedisplayshortskip}{0.05 in}
  \setlength{\belowdisplayshortskip}{0.05 in}}
\appto{\normalsize}{\zerodisplayskips}
\appto{\small}{\zerodisplayskips}
\appto{\footnotesize}{\zerodisplayskips}

\setbeamerfont{caption}{size=\scriptsize}
\setbeamerfont{footnote}{size=\scriptsize}


% include pdf version 1.6
%\pdfoptionpdfminorversion=7 


\usepackage{kotex}
\usepackage{textcomp} % this is to use Korean currency, Won.

\usepackage{scrextend}

\usepackage{fancybox}     
\usepackage{booktabs}              
\usepackage{multirow}

\newcolumntype{?}{!{\vrule width 1pt}}

%%%% testing environments

%\usepackage{thmtools}
%\usepackage{boiboites} % this requires boiboites.sty in the folder
\usepackage{eshinbox}

\newcounter{thm}
\numberwithin{thm}{section}

\newboxedtheorem[boxcolor=black, background=gray!5, titlebackground=gray!20,
titleboxcolor = black]{thm}{Theorem}{thm}

\newboxedtheorem[boxcolor=black, background=gray!5, titlebackground=gray!20,
titleboxcolor = black]{prp}{Proposition}{thm}

%%%%%%%%%%%%%%%%%%%%%%%%%%%%%%%% START %%%%%%%%%%%%%%%%%%%%%%%%%%%%%%%%%%%%%%

\title[Short title of the talk]{Rationality and Preference Aggregation of Group Decision under Risk}
\author{Syngjoo Choi (SNU), Booyuel Kim (KDIS), Minseon Park (Wisconsin), Yoonsoo Park (KDI), Euncheol Shin (KHU)}
\institute[]
{Presenter: Euncheol Shin (Kyung Hee University) \\
\medskip
{\tt{eshin.econ@khu.ac.kr}}
}
\date{%June, 2017
\today %will show current date.
%324 Alternatively, you can specify a date.
}

%
\begin{document}
%
\begin{frame}
\titlepage
\end{frame}

%%%%%%%%%%%%%%%%%%%%%%%% new frame %%%%%%%%%%%%%%%%%%%%%%%%

\begin{frame}{Introduction}

\begin{itemize}

\item In various contexts, many important decisions are made by groups.
\mk

\item Heterogeneity in various dimensions:
\sk
	\begin{itemize}
	
	\item Risk preference: risk assessment in environmental policy-making in committees
	\sk
	
	\item Time preference: household savings and consumption decisions.
	\sk

	\item Rationality
	
	\end{itemize}
	\mk
	
\item It is important to understand how individual heterogeneity in a collective influences final outcomes.

\end{itemize}

\end{frame}

%%%%%%%%%%%%%%%%%%%%%%%% new frame %%%%%%%%%%%%%%%%%%%%%%%%

\begin{frame}{Introduction: Research Questions}

\begin{itemize}

\item[1.] Rationality extension:
	\sk
	\begin{itemize}
	
	\item If each individual's choices are consistent with a utility maximization model, do a group's choices also tend to be?	
	
	\end{itemize}
	\mk

\item[2] Risk preference aggregation:
	\sk
	\begin{itemize}
	
	\item Are individual's risk preferences reflected into that of a group?
	
	\end{itemize}
	\mk
	
\item[3.] Efficiency:
	\sk
	\begin{itemize}
	
	\item Are group's choices Pareto efficient? 
	\sk
	
	\item How is the efficiency related to group members' rationality and preferences?
	
	\end{itemize}
	
\end{itemize}

\end{frame}

%%%%%%%%%%%%%%%%%%%%%%%% new frame %%%%%%%%%%%%%%%%%%%%%%%%

\begin{frame}{Introduction: Why Experiments?}

\begin{itemize}

\item The laboratory experiment can be stripped of many confounding factors, and decisions can be observed in a highly \textcolor{tred}{\textbf{controlled environment}}.
\mk

\item By applying the revealed preference theory, we can directly measure rationality and risk preference without noise.
\mk
%\pause

\item Beyond measuring rationality and risk preference, we can investigate how these are \textcolor{tred}{\textbf{aggregated}} in collective decisions.
\mk

\item We finally analyze an efficiency of group decisions as a function of members' rationality and preference. 

\end{itemize}

\end{frame}

%%%%%%%%%%%%%%%%%%%%%%%% new frame %%%%%%%%%%%%%%%%%%%%%%%%

\begin{frame}{Related Literature}

\begin{itemize}

\item[1.] Preference aggregation
	\sk
	\begin{itemize}
	
	\item 
	
	\end{itemize}
	\mk

\item[2] Testable implication
	\sk
	\begin{itemize}
	
	\item 
	
	\end{itemize}
	\mk
	
\item[3.] Intra-household bargaining
	\sk
	\begin{itemize}
	
	\item 
	
	\end{itemize}
	
\end{itemize}

\end{frame}

%%%%%%%%%%%%%%%%%%%%%%%% new frame %%%%%%%%%%%%%%%%%%%%%%%%

\begin{frame}[noframenumbering]{}

\center
\textbf{Experimental Design and Subjects}

\end{frame}

%%%%%%%%%%%%%%%%%%%%%%%% new frame %%%%%%%%%%%%%%%%%%%%%%%%

\begin{frame}{Experimental Design}
\nk

\begin{figure}
\center
\large 
\begin{tikzpicture}

% axes
\draw[->, thick] (0,0) -- (7,0) node[anchor = west] {$x_r$};
\draw[->, thick] (0,0) -- (0,7) node[anchor = east] {$x_b$};
		
% budget lines
\draw[thick] (6,0) -- (0,3);

% 45 degree line
\draw[dotted, thick] (0,0) -- (4,4) node[anchor = south west] {$x_r = x_b$};
\draw (0.8, 0) node[anchor = south] {$45^\circ$};

% prices
\draw[->, thick] (6,0) -- (6+1/2, 1) node[anchor = west] {$p = (p_r, p_b)$};

% bundles
\draw (2, 2) node {\textbullet};
\draw (2, 2.1) node[anchor = south] {$x$};

\draw (3, 6-3/2*3) node {\textbullet};
\draw (3, 6-3/2*3) node[anchor = north east] {$x'$};

\draw (6, 0) node {\textbullet};
\draw (6, - 0.6) node[anchor = south] {$x''$};

% texts
\visible<1>{
\small{
\draw (3, 7) node[anchor = west] {Two equally likely states: $\textcolor{tred}{R}$ and $\textcolor{tblue}{B}$.};

\draw (3, 6.5) node[anchor = west] {There are two associated Arrow securities.};

\draw (3, 6) node[anchor = west] {$x_r$ is the demand for the security that pays off in state $\textcolor{tred}{R}$.};

\draw (3, 5.5) node[anchor = west] {$x_b$ is the demand for the security that pays off in state $\textcolor{tblue}{B}$.};

\draw (3, 5) node[anchor = west] {Budget constraint: $p_r x_r + p_b x_b = 1$.};
}
}

\visible<2>{
\small{
\draw (3, 7) node[anchor = west] {In this example, $p_b > p_r$.};

\draw (3, 6.5) node[anchor = west] {Risk neutral agent will choose $x''$.};

\draw (3, 6) node[anchor = west] {Extremely risk averse agent will choose $x$.};

\draw (3, 5.5) node[anchor = west] {Intermediately risk averse agent will choose $x'$.};

%\draw (3, 5) node[anchor = west] {$r = \frac{x_r}{x_r + x_b}$};
}
}

\end{tikzpicture}
\end{figure}

\end{frame}

%%%%%%%%%%%%%%%%%%%%%%%% new frame %%%%%%%%%%%%%%%%%%%%%%%%

\begin{frame}{Procedure and Subjects}

\begin{itemize}

\item We conducted the experiment in 12 middle schools in Daegu.
\mk

\item The number of students: 1572.
\mk 

\item The number of groups: 786.
\mk

\item The instructions were read by an experiment in each classroom.
\mk
	
\item Each subject participate in two sessions: individual and group decisions.
\mk
	
\item Each session consisted of 18 independent decision rounds.
\mk

\item Each round started by having the computer select a budget line randomly from the set of lines that intersect at least one axis at or above $300$ KRW or below $1500$ KRW.

\end{itemize}

\end{frame}

%%%%%%%%%%%%%%%%%%%%%%%% new frame %%%%%%%%%%%%%%%%%%%%%%%%

\begin{frame}{Procedure and Subjects}

\begin{itemize}

\item At the end of each round, the computer randomly selected one of the two states ($\textcolor{tred}{R}$ and $\textcolor{tblue}{B}$).
\mk

\item Subjects were not informed of the state that was actually selected at the end of each round.
\mk

\item Each subject was paid for he/she earned in a randomly selected round.
\mk

\item After every student finishes the first session, the second session begins.
\mk

\item Two students in the same classroom were randomly matched.
\mk

\item A student moved to the other partner's desk and make collective decisions by sharing the computer.
\mk

\item We allowed students to discuss how to make decisions for 1 min before starting the second session.

\end{itemize}

\end{frame}

%%%%%%%%%%%%%%%%%%%%%%%% new frame %%%%%%%%%%%%%%%%%%%%%%%%

\begin{frame}{Experimental Design}

\begin{figure}
\center
\includegraphics[width = 4.0 in]{figures/pblrealshot}
\end{figure}

\end{frame}

%%%%%%%%%%%%%%%%%%%%%%%% new frame %%%%%%%%%%%%%%%%%%%%%%%%

\begin{frame}{Experimental Design: Screenshot}

\begin{figure}
\center
\includegraphics[width = 3.8 in]{figures/expshot}
\end{figure}

\end{frame}

%%%%%%%%%%%%%%%%%%%%%%%% new frame %%%%%%%%%%%%%%%%%%%%%%%%

\begin{frame}[noframenumbering]{Experimental Design: Example}

\begin{figure}
\center
%\includegraphics[width = 4.5 in]<1>{figures/Figure_Choice_Example1}
\includegraphics[width = 4.5 in]<1>{figures/Figure_Choice_Example2}
\end{figure}

\end{frame}

%%%%%%%%%%%%%%%%%%%%%%%% new frame %%%%%%%%%%%%%%%%%%%%%%%%

\begin{frame}[noframenumbering]{}

\center
\textbf{Result 1: Rationality Extension}

\end{frame}

%%%%%%%%%%%%%%%%%%%%%%%% new frame %%%%%%%%%%%%%%%%%%%%%%%%

\begin{frame}{Measurement: Afriat's Efficiency Index (a.k.a. CCEI)}

\begin{figure}
\center
\large 
\begin{tikzpicture}

% axes
\draw[->, thick] (0,0) -- (7,0) node[anchor = west] {};
\draw[->, thick] (0,0) -- (0,7) node[anchor = east] {};
		
% budget lines
\draw[thick] (6,0) -- (0,4);
\draw[thick] (0,6) -- (4,0);

% bundles
\draw (3.3, 6-3/2*3.3) node {\textbullet};
\draw (3.3, 6-3/2*3.3) node[anchor = north east] {$x'$};

\draw (6-3/2*3, 3) node {\textbullet};
\draw (6-3/2*3, 3) node[anchor = north east] {$x$};

% rationalization
\draw[red, ultra thick] (0,6-3/2*0.5) -- (4-0.5,0);

% texts
\small{
\draw (3, 7) node[anchor = west] {If there are two goods, a choice dataset satisfies the GARP};

\draw (3, 6.5) node[anchor = west] {if and only if it satisfies the WARP.};

\draw (3, 5.5) node[anchor = west] {$x'$ and $x$ violate GARP.};

\draw (3, 4.5) node[anchor = west] {If the budget line for $x'$ is deflated, then GARP is satisfied.};

\draw (3, 4) node[anchor = west] {Choose $e_v$ for each violation $v$.};

\draw (3, 3) node[anchor = west] {CCEI is defined as the supremum over all the numbers $e_v$'s.};

}

\end{tikzpicture}
\end{figure}

\end{frame}

%%%%%%%%%%%%%%%%%%%%%%%% new frame %%%%%%%%%%%%%%%%%%%%%%%%

\begin{frame}
\frametitle{Measurement: Afriat's Efficiency Index (a.k.a. CCEI)}

\begin{itemize}

\item By definition, CCEI $\in [0,1]$.
\mk

\item The \textcolor{tred}{bigger} CCEI is, the \textcolor{tred}{less} severe violation of GARP.
\mk

\item Basic statistics of individual CCEI:
	\sk
	\begin{itemize}
	
	\item Average: 0.897 (0.136)
	\sk
	
	\item Quantiles: 0.838, 0.953, 1.
	
	\end{itemize}
	\mk 
	
\item Basic statistics of collective CCEI:
	\sk
	\begin{itemize}
	
	\item Average: 0.897 (0.136)
	\sk
	
	\item Quantiles: 0.868, 0.981, 1.
	
	\end{itemize}

\end{itemize}

\end{frame}

%%%%%%%%%%%%%%%%%%%%%%%% new frame %%%%%%%%%%%%%%%%%%%%%%%%

\begin{frame}{Rationality Extension: Research Question}

\center
Individual Rationality $\uparrow$ $\Rightarrow$ Collective Rationality $\uparrow$?

\end{frame}

%%%%%%%%%%%%%%%%%%%%%%%% new frame %%%%%%%%%%%%%%%%%%%%%%%%

\begin{frame}{Rationality Extension: First-Order Stochastic Dominance}

\begin{figure}
\center
\includegraphics[width = 4.5 in]{figures/Figure1_CCEI_90}
%\caption{Computerized experiment in a classroom}
\end{figure}

\end{frame}

%%%%%%%%%%%%%%%%%%%%%%%% new frame %%%%%%%%%%%%%%%%%%%%%%%%

\begin{frame}
\frametitle{Rationality Extension: First-Order Stochastic Dominance}

\begin{itemize}

\item We do a series of Kolmogorov-Smirnov tests:
	\sk
	\begin{align*}
	H_0: F_{\text{group $i$}} (X) = F_{\text{group $j$}} (X) \quad \text{for all values of $X$}.
	\end{align*}
\sk \nk

\item (Low, Low) v.s. (High, High): 0.17
\sk
	\begin{itemize}
	
	\item The corresponding p-value is 0.01.

	\end{itemize}
	\mk
	
\item (Low, High) v.s. (High, High): 0.21
\sk
	\begin{itemize}
	
	\item The corresponding p-value is 0.00.

	\end{itemize}
	\mk
	\pause
	
\item Our results are robust with respect to 
	\sk
	\begin{itemize}
	\item another rationality measure (Varian's efficiency index)
	\sk
	
	\item cutoff values (0.99 or 0.95).
	
	\end{itemize}

\end{itemize}

\end{frame}

%%%%%%%%%%%%%%%%%%%%%%%% new frame %%%%%%%%%%%%%%%%%%%%%%%%

\begin{frame}{Rationality Extension: Econometric Analysis}

\vspace{0.10 in}
%\hspace{-0.15 in}
\centering \changefontsizes{8pt}
\begin{tabular}{l  lc lc lc}
\toprule
\multirow{2}{*}{\textbf{Collective CCEI}}	& & \multicolumn{5}{c}{\textbf{Coefficient}}  \\
		        \cline{3-7}
																& & Model 1 & & Model 2 & & Model 3 \\
\hline
\hline
\multirow{2}{*}{CCEI\underline{{ }{ }}Max}  & &  $0.368^{***}$  & & $0.327^{***}$  & & $0.302^{***}$   \\
                                                                        & & $(0.083)$         & & $(0.074)$          & & $(0.089)$          \\
\multirow{2}{*}{CCEI\underline{{ }{ }}Distance}  & & $-0.277^{***}$ & & $-0.250^{***}$  & &  $-0.233^{***}$  \\
                                                                                & & $(0.056)$        & & $(0.053)$           & & $(0.058)$       \\
\hline
\multirow{2}{*}{Risk\underline{{ }{ }}Aversion\underline{{ }{ }}Max}  & &      & & $-0.189^{***}$  & &  $-0.172^{**}$    \\
                                                                                                                 & &      & & $(0.056)$      & & $(0.070)$    \\
\multirow{2}{*}{Risk\underline{{ }{ }}Aversion\underline{{ }{ }}Distance}  & &       & & $0.087^{*}$     & &  $0.093^{*}$   \\
                             																			         & &      & & $(0.048)$   & & $(0.055)$   \\
\hline
\multirow{2}{*}{Math\underline{{ }{ }}Score\underline{{ }{ }}Max}  & &     & &     & &  $0.012^{**}$  \\
                                                                                                             & &     & &     & &  $(0.005)$     \\
\multirow{2}{*}{Math\underline{{ }{ }}Distance}  & &        & &         & &  $-0.010^{**}$      \\
                                                						        & &          & &        & &  $(0.005)$     \\
\multirow{2}{*}{Constant}      & & $0.582^{***}$   & & $0.679^{***}$    & & $0.664^{***}$  \\
                                                 	  & &  $(0.077)$        & &  $(0.070)$         & &  $(0.084)$  \\
\hline
\hline
Class Fixed Effect                  & & Yes         & & Yes         & & Yes            \\
Individual Characteristics      & & No         & & No         & & Yes         \\
School Characteristics           & & No         & &  No         & & Yes          \\
Friendship                              & & No         & &  No         & & Yes        \\
Observations                  & & $786$ & & $786$ & & $786$    \\
$R$-squared                  & & $0.200$  & & $0.212$  & & $0.235$  \\
\bottomrule 
\end{tabular}

\end{frame}

%%%%%%%%%%%%%%%%%%%%%%%% new frame %%%%%%%%%%%%%%%%%%%%%%%%

\begin{frame}[noframenumbering]{}

\center
\textbf{Result 2: Preference Aggregation}

\end{frame}

%%%%%%%%%%%%%%%%%%%%%%%% new frame %%%%%%%%%%%%%%%%%%%%%%%%

\begin{frame}
\frametitle{Measurement: A Parametric Measure}

\begin{itemize}

\item We consider a constant absolute risk aversion (CARA) utility function over outcomes:
	\begin{align*}
	u(x) = - \frac{ e^{- \rho x} }{ \rho }.
	\end{align*}
	\sk \nk

\item We consider two different types of utility function over lotteries:
	\sk
	\begin{itemize}
	\item Expected utility (EU) by von Neumann and Morgenstern (1953):
		\begin{align*}
		U(Z) = \frac{1}{2} u(x_{\text{max}}) + \frac{1}{2} u(x_{\text{min}}).
		\end{align*}
	\sk
	
	\item Disappointment aversion utility (DAU) by Gul (1991):
		\begin{align*}
		U(Z) = \alpha u(x_{\text{max}}) + (1-\alpha) u(x_{\text{min}}) \quad \text{where $\alpha = \frac{1}{2+\beta}$}
		\end{align*}
		\begin{itemize}
		\item $\beta > 0$: disappointment aversion
		\sk
		
		\item $-1 < \beta <0$: elation seeking.
		
		\end{itemize}
		
	
	\end{itemize}
	\mk

\item We estimate $\rho$ and $\beta$ simultaneously by using NLLS.

\end{itemize}

\end{frame}

%%%%%%%%%%%%%%%%%%%%%%%% new frame %%%%%%%%%%%%%%%%%%%%%%%%

\begin{frame}{Measurement: Indifference Curves}
\nk

\begin{figure}
\center
\large 
\begin{tikzpicture}

% axes
\draw[->, thick] (0,0) -- (7,0) node[anchor = west] {$x_r$};
\draw[->, thick] (0,0) -- (0,7) node[anchor = east] {$x_b$};
		
% budget lines
\draw[thick] (6,0) -- (0,3);

% 45 degree line
\draw[dotted, thick] (0,0) -- (4,4) node[anchor = south west] {};
\draw (0.8, 0) node[anchor = south] {$45^\circ$};

\visible<1-1>{
% indifference curve
\draw[thick, tred] (2.3, 2.3) to [out=-45, in=170] (6,0.4) node[anchor = south west] {};
\draw[thick, tred] (2.3, 2.3) to [out=135, in=-80] (0.4,6) node[anchor = south west] {};
% choice
\draw (4, 1) node {\textbullet};
\draw (4, 1) node[anchor = south west] {$x_{\text{EU}}$};
}

\visible<2-2>{
% indifference curve
\draw[thick, tblue] (2.3, 2.3) to [out=-50, in=170] (6,0.5) node[anchor = south west] {};
\draw[thick, tblue] (2.3, 2.3) to [out=140, in=-80] (0.5,6) node[anchor = south west] {};
% choice
\draw (3.7, 1.15) node {\textbullet};
\draw (3.8, 1.15) node[anchor = north east] {$x_{\text{RDU}}$};
}

\visible<3->{
% choice
\draw (3.7, 1.15) node {\textbullet};
\draw (3.5, 1.15) node[anchor = south west] {$x = (x_{\text{cheap}} , x_{\text{expensive}} )$};
\draw[->, thick] (6,0) -- (6+1/2, 1) node[anchor = west] {$p = (p_r, p_b)$};
}

% texts
\visible<1->{
\small{
\draw (3, 7) node[anchor = west] {We assume a CARA utility function: $u(x) = - e^{-\rho x} / \rho$.};

\draw (3, 6.5) node[anchor = west] {The red line is an indifference curve for the EU.};
}
}

\visible<2->{
\small{
\draw (3, 6) node[anchor = west] {The blue curve is an indifference curve for the RDU.};

\draw (3, 5.5) node[anchor = west] {$\rho$ determines the curvature of the curve.};

\draw (3, 5) node[anchor = west] {$\alpha$ determines the shape of the kink.};
}
}

\visible<3->{
\small{
\draw (3, 4.5) node[anchor = west] {We can simply measure the risk preference by a ratio:};

\draw (4.5, 3.8) node[anchor = west] {risk aversion = $\frac{x_{\text{expensive}} }{ x_{\text{expensive}} + x_{\text{cheap}} }$};
}
}

\end{tikzpicture}
\end{figure}

\end{frame}

%%%%%%%%%%%%%%%%%%%%%%%% new frame %%%%%%%%%%%%%%%%%%%%%%%%

\begin{frame}
\frametitle{Measurement: Risk Preferences}

\begin{itemize}

\item The degree of risk aversion is measured by both parametric and non-paremetric methods.
\mk

\item The \textcolor{tred}{bigger} premium/ratio is, the \textcolor{tred}{higher} risk aversion.
\mk

\item Basic statistics of individual risk preference:
	\sk
	\begin{itemize}
	
	\item Risk premium: 0.220	 (0.310)
	\sk
	
	\item Ratio: 0.324	(0.132).
	
	\end{itemize}
	\mk
	
\item Basic statistics of collective risk preference:
	\sk
	\begin{itemize}
	
	\item Risk premium: 0.258  (0.371)
	\sk
	
	\item Ratio: 0.298	(0.139).
	
	\end{itemize}

\end{itemize}

\end{frame}

%%%%%%%%%%%%%%%%%%%%%%%% new frame %%%%%%%%%%%%%%%%%%%%%%%%

\begin{frame}{Risk Preference Aggregation: Research Question}

\center
Individual risk aversion $\uparrow$ $\Rightarrow$ Collective risk aversion $\uparrow$?

\end{frame}

%%%%%%%%%%%%%%%%%%%%%%%% new frame %%%%%%%%%%%%%%%%%%%%%%%%

\begin{frame}{Risk Preference Aggregation: FOSD by Premium}

\begin{figure}
\center
\includegraphics[width = 4.5 in]{figures/Figure2B_riskpremium}
%\caption{Computerized experiment in a classroom}
\end{figure}

\end{frame}

%%%%%%%%%%%%%%%%%%%%%%%% new frame %%%%%%%%%%%%%%%%%%%%%%%%

\begin{frame}{Risk Preference Aggregation: FOSD by Relative Ratio}

\begin{figure}
\center
\includegraphics[width = 4.5 in]{figures/Figure2A_riskaversion}
%\caption{Computerized experiment in a classroom}
\end{figure}

\end{frame}

%%%%%%%%%%%%%%%%%%%%%%%% new frame %%%%%%%%%%%%%%%%%%%%%%%%

\begin{frame}{Risk Preference Aggregation: Econometric Analysis}

\vspace{0.10 in}
%\hspace{-0.15 in}
\centering \changefontsizes{8pt}
\begin{tabular}{l  lc lc lc}
\toprule
\multirow{2}{*}{\textbf{Collective Risk Aversion}}	& & \multicolumn{5}{c}{\textbf{Coefficient}}  \\
		        \cline{3-7}
																& & Model 1 & & Model 2 & & Model 3 \\
\hline
\hline
\multirow{2}{*}{Risk\underline{{ }{ }}Aversion\underline{{ }{ }}Max}  & &  $0.792^{***}$  & & $0.808^{***}$  & & $0.759^{***}$ \\
                                                                                                                 & &  $(0.066)	$  & & $(0.063)$      & & $(0.073)$    \\
\multirow{2}{*}{Risk\underline{{ }{ }}Aversion\underline{{ }{ }}Distance}  & &  $-0.421^{***}$  & & $-0.432^{***}$   & &  $-0.434^{***}$   \\
                             																			         & &  $(0.053)$  & & $(0.053)$   & & $(0.062)$   \\
\hline
\multirow{2}{*}{CCEI\underline{{ }{ }}Max}  & &      & & $0.165^{**}$  & & $0.196^{***}$   \\
                                                                        & &       & & $(0.064)$          & & $(0.071)$          \\
\multirow{2}{*}{CCEI\underline{{ }{ }}Distance}  & &    & & $0.014$  & &  $0.040$  \\
                                                                                & &    & & $(0.033)$           & & $(0.045)$       \\
\hline
\multirow{2}{*}{Math\underline{{ }{ }}Score\underline{{ }{ }}Max}  & &     & &     & &  $0.000$  \\
                                                                                                             & &     & &     & &  $(0.005)$     \\
\multirow{2}{*}{Math\underline{{ }{ }}Distance}  & &        & &         & &  $0.005$      \\
                                                						        & &          & &        & &  $(0.004)$     \\
\multirow{2}{*}{Constant}      & & $0.004$   & & $-0.156^{**}$    & & $-0.175^{**}$  \\
                                                 	  & &  $(0.027)$        & &  $(0.063)$         & &  $(0.077)$  \\
\hline
\hline
Class Fixed Effect                  & & Yes         & & Yes         & & Yes            \\
Individual Characteristics      & & No         & & No         & & Yes         \\
School Characteristics           & & No         & &  No         & & Yes          \\
Friendship                              & & No         & &  No         & & Yes        \\
Observations                  & & $786$ & & $786$ & & $786$    \\
$R$-squared                  & & $0.372$  & & $0.378$  & & $0.382$  \\
\bottomrule 
\end{tabular}

\end{frame}

%%%%%%%%%%%%%%%%%%%%%%%% new frame %%%%%%%%%%%%%%%%%%%%%%%%

\begin{frame}[noframenumbering]{}

\center
\textbf{Result 3: Efficiency}

\end{frame}

%%%%%%%%%%%%%%%%%%%%%%%% new frame %%%%%%%%%%%%%%%%%%%%%%%%

\begin{frame}{Efficiency: Measurement}

\begin{itemize}

\item \textbf{Claim:} A group's choice $x_c$ is \textbf{Pareto efficient} if and only if it is between the members' optimal choices.
\mk

\item  We measure the group inefficiency as the average utility loss:
	\sk
	\begin{align*}
	\text{Inefficiency}_{g} = \frac{1}{18} \sum_{k = 1}^{18} \frac{1}{2} \sum_{i = 1}^2 \frac{u_i(x_{ikb}) - u_i(x_{ck})}{u_i(x_{ikb}) - u_i(x_{ike})}.
	\end{align*}
	\sk
	where
	\sk
	\begin{itemize}
	
	\item $x_{ck}$: group choice in $k$-th round
	\sk
	
	\item $x_{ikb}$: member $i$'s best choice in $k$-th round
	\sk
	
	\item $x_{ikw}$: member $i$'s worst choice in $k$-th round.
	
	\end{itemize}
	\mk
	
\item By definition, $\text{Loss}_{g} \in [0,1]$.

\end{itemize}

\end{frame}

%%%%%%%%%%%%%%%%%%%%%%%% new frame %%%%%%%%%%%%%%%%%%%%%%%%

\begin{frame}{Efficiency: Distribution of Group Inefficiency}

\begin{figure}
\center
\includegraphics[width = 4.5 in]{figures/Figure3A_ULoss}
%\caption{Computerized experiment in a classroom}
\end{figure}

\end{frame}

%%%%%%%%%%%%%%%%%%%%%%%% new frame %%%%%%%%%%%%%%%%%%%%%%%%

\begin{frame}{Efficiency: Research Question}

\center
How is the efficiency related to individual rationality and risk preference?

\end{frame}

%%%%%%%%%%%%%%%%%%%%%%%% new frame %%%%%%%%%%%%%%%%%%%%%%%%

\begin{frame}{Efficiency: FOSD by Group Rationality}

\begin{figure}
\center
\includegraphics[width = 4.5 in]{figures/Figure4A_ULoss_ccei}
%\caption{Computerized experiment in a classroom}
\end{figure}

\end{frame}

%%%%%%%%%%%%%%%%%%%%%%%% new frame %%%%%%%%%%%%%%%%%%%%%%%%

\begin{frame}{Efficiency: Econometric Analysis}

\vspace{0.045 in}
%\hspace{-0.15 in}
\centering \changefontsizes{8pt}
\begin{tabular}{l  lc lc lc}
\toprule
\multirow{2}{*}{\textbf{Group Inefficiency}}	& & \multicolumn{5}{c}{\textbf{Coefficient}}  \\
		        \cline{3-7}
																& & Model 1 & & Model 2 & & Model 3 \\
\hline
\hline
\multirow{2}{*}{CCEI\underline{{ }{ }}Group}  & &  $-0.571^{***}$  & & $-0.503^{***}$  & & $-0.527^{***}$   \\
                                                                        & & $(0.042)$         & & $(0.039)$          & & $(0.043)$          \\
\hline
\multirow{2}{*}{CCEI\underline{{ }{ }}Max}  & &      & & $-0.296^{***}$  & & $-0.242^{***}$   \\
                                                                        & &       & & $(0.045)$          & & $(0.051)$          \\
\multirow{2}{*}{CCEI\underline{{ }{ }}Distance}  & &         & & $0.165^{***}$  & &  $0.178^{***}$  \\
                                                                                & &        & & $(0.042)$           & & $(0.052)$       \\
\hline
\multirow{2}{*}{Risk\underline{{ }{ }}Aversion\underline{{ }{ }}Max}  & &      & & $-0.009$  & &  $0.023$    \\
                                                                                                                 & &      & & $(0.056)$      & & $(0.063)$    \\
\multirow{2}{*}{Risk\underline{{ }{ }}Aversion\underline{{ }{ }}Distance}  & &       & & $-0.057^{*}$     & &  $-0.073^{*}$   \\
                             																			         & &      & & $(0.031)$   & & $(0.040)$   \\
\hline
\multirow{2}{*}{Math\underline{{ }{ }}Score\underline{{ }{ }}Max}  & &     & &     & &  $0.002$  \\
                                                                                                             & &     & &     & &  $(0.005)$     \\
\multirow{2}{*}{Math\underline{{ }{ }}Distance}  & &        & &         & &  $-0.004$      \\
                                                						        & &          & &        & &  $(0.003)$     \\
\multirow{2}{*}{Constant}      & & $0.651^{***}$   & & $0.866^{***}$    & & $0.807^{***}$  \\
                                                 	  & &  $(0.038)$        & &  $(0.048)$         & &  $(0.061)$  \\
\hline
\hline
Class Fixed Effect                  & & Yes         & & Yes         & & Yes            \\
Individual Characteristics      & & No         & & No         & & Yes         \\
School Characteristics           & & No         & &  No         & & Yes          \\
Friendship                              & & No         & &  No         & & Yes        \\
Observations                  & & $786$ & & $786$ & & $786$    \\
$R$-squared                  & & $0.442$  & & $0.487$  & & $0.497$  \\
\bottomrule 
\end{tabular}

\end{frame}

%%%%%%%%%%%%%%%%%%%%%%%% new frame %%%%%%%%%%%%%%%%%%%%%%%%

\begin{frame}{Conclusion}

\begin{itemize}

\item In this paper, we measure rationality and risk preference in both individual and group levels.
\mk

\item More rational individuals are more likely to make rational choices.
\mk

\item More risk-averse individuals are more likely to make risk-averse choices.
\mk

\item More rational groups choices are more likely to efficient decisions.
\mk

\item More preference-aligned individuals need not make efficient decisions.
	
\end{itemize}

\end{frame}

\appendix

%%%%%%%%%%%%%%%%%%%%%%%% new frame %%%%%%%%%%%%%%%%%%%%%%%%

\begin{frame}{}

\center
\textbf{Robustness}

\end{frame}

%%%%%%%%%%%%%%%%%%%%%%%% new frame %%%%%%%%%%%%%%%%%%%%%%%%

\begin{frame}{Varian's Efficiency Index}

\begin{itemize}

\item 

\end{itemize}

\end{frame}

%%%%%%%%%%%%%%%%%%%%%%%% new frame %%%%%%%%%%%%%%%%%%%%%%%%

\begin{frame}{Introduction: Risk Perception}

\begin{itemize}

\item[1.] Genetics: White males wre more likely to perceive risks as being smaller (Bickerstaff, 2004; Flynn et al., 1994).
\mk

\item[2.] Psychology: 
\sk
	\begin{itemize}
	
	\item availability bias 
	\sk
	
	\item experience
	\end{itemize}
	\mk
	
\item[3.] Knowledge and Information: public v.s. experts
\mk
	
\item[4.] Other factors
\sk
	\begin{itemize}
	
	\item culture: Chinese individuals are significantly less risk-averse than individuals from Western countries when making financial decisions (Weber and Hsee, 1998).
	\sk
	
	\item power: high-power groups adopt a more positive attitude toward potential risks (Anderson and Galinsky, 2006; Magee et al., 2007; Geng et al, 2018).
	\sk
	
	\item social inequality: Sweden people with foreign backgrounds did perceive risks as higher than native people, but no difference between men and women.
	
	\end{itemize}

\end{itemize}

\end{frame}

%%%%%%%%%%%%%%%%%%%%%%%% new frame %%%%%%%%%%%%%%%%%%%%%%%%

\begin{frame}{Rank-Dependent Utility Function}

\begin{itemize}

\item A functional form:
\mk

\item Properties:
	\sk
	\begin{itemize}
	
	\item An indifference curve is only partially convex.
	\sk
	
	\item The slope of an indifference curve is calculated as
	\begin{align*}
	\frac{dy}{dx} = 
	\end{align*}
	
	\end{itemize}
	\mk

\item With a CARA (or CRRA) utility function, it can rationalize all the choices except the corner choices.
\mk

\item When there are only two states, it is equivalent to the disappointment-aversion utility function by Gul (1991).
\mk

\item Of course, this utility function is aligned with the first-order stochastic dominance relationships between lotteries.

\end{itemize}

\end{frame}

%%%%%%%%%%%%%%%%%%%%%%%% new frame %%%%%%%%%%%%%%%%%%%%%%%%

\begin{frame}{Pareto Efficiency: Measurement}

\begin{itemize}

\item For given budget set, let $x_1^*$ and $x_2^*$ be the optimal portfolio choice of agent 1 and agent 2, respectively.
\mk

\item \textbf{Claim:} A group choice $x_c$ is \textbf{Pareto efficient} if and only if $x_c \in [x_1^*, x_2^*]$.

\end{itemize}

\end{frame}

%%%%%%%%%%%%%%%%%%%%%%%% new frame %%%%%%%%%%%%%%%%%%%%%%%%

%\begin{frame}{Introduction to Experimental Economics}

%\center
%\includegraphics[height = 3 in]<1>{pfigs/dan}
%\includegraphics[height = 3 in]<2>{pfigs/roth}
%\includegraphics[height = 3 in]<3>{pfigs/thaler1}
%\includegraphics[height = 3 in]<4>{pfigs/thaler2}

%\end{frame}

\end{document}


%%%%%% Inventory

%\center
%\includegraphics[width = 1.8 in]{textbook}
